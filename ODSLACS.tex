

\chapter{MULTI-LAYERED LASER-ASSISTED COLD SPRAY DEPOSITION OF OXIDE DISPERSION STRENGTHENED Fe\textsubscript{91}Ni\textsubscript{8}Zr\textsubscript{1}}

 

Note: This paper will soon be under review for publication in a peer-reviewed journal.




\section*{Abstract}

	Six-sequential layer deposits of an oxide dispersion strengthened (ODS) Fe\textsubscript{91}Ni\textsubscript{8}Zr\textsubscript{1} alloy were laser assisted cold sprayed at 650 \celsius{} and 950 \celsius{}. The ODS alloy powders used for deposition were fabricated by Zoz ball milling of constituent elemental powders where intrinsic oxygen contaminates in the powders reacted to form nano-scale zirconia particles. The laser provides a local heating of the substrate, which thermally softens it, to help increase the deposition efficiency. This is particularly important when depositing ODS powders onto the prior deposited ODS layers, where some amount of increased plastic deformation is required for the particles to adhere. At 650 \celsius{}, the ODS alloy stayed ferritic during deposition, with this phase’s associated limited ductility resulting in a deposition efficiency of 7.3 $\%$ . The lower deposition efficiency resulted in a peening or tamping effect to the desposit, where the grains at the substrate/deposit interface were equaxied and refined suggesting a dynamic recrystallization event. At 950 \celsius{}, the ODS alloy was in the austenitic phase field during deposition which increased its ductility and the deposition efficiency increased to 32.4 $\%$ . With the increase in deposition temperature, the resulting deposited grains grew resulting in a change in hardness from 598 $ \pm $  56 (650 \celsius{}) to 293 $ \pm $  38 (950 \celsius{}). These results demonstrate the advantages of laser assisted deposition for relatively difficult-to-spray (high hardness and strength) materials, albeit with the associated changes in grain sizes and corresponding changes in hardness. 
	

	\section*{Introduction}

	Cold spray (CS) deposition is an additive manufacturing and repair technology \cite{RN3390}. It is based upon ejecting powder particles at supersonic velocities onto a substrate whereupon the deposited material is built layer-by-layer. It has shown promise in the deposition of several elemental and alloy systems including light-weight Al \cite{RN641,RN626}, Cu \cite{RN1172,RN817}, and Ti \cite{RN2281,RN678} alloys. During CS deposition, at high velocities, the particles experience significant strain rates upon impact that are on the order of 10\textsuperscript{9} s\textsuperscript{-1} \cite{RN1687}, which requires the deposited material to undergo significant plasticity in order to adhere to the substrate and other particles through mechanical interlocking and metallurgical bonding \cite{RN486}. This poses a significant challenge for materials that have limited slip systems, such as hexagonal close packed (HCP) materials, as well as those with significant strengthening secondary phases, such as oxide dispersion strengthened (ODS) alloys. ODS alloys are a particularly interesting class of materials because of their potential uses in nuclear energy, where the nanoscale oxides create numerous interfaces that provide sinks for irradiation-created defects \cite{RN47,RN353}. ODS alloys are also candidate materials for ballistic resistance applications because of their high-strain rate compressive strengths that exceed 2 GPa \cite{RN267}, which makes them ideal for such applications while a difficult material for CS deposition. Fine-grained steel with a composition of a common ODS steel has been deposited onto an Al substrate \cite{RN378}, but CS deposition of ball-milled ODS powder onto steel has proven to be challenging \cite{RN383}.



	A proposed means to overcome these challenges for depositing the aforementioned ‘difficult materials’ is laser assisted cold spray (LACS) \cite{RN173,RN3366}. Here, a laser locally increases the substrate surface temperature directly under where the cold spray deposit occurs. By elevating the surface temperature, the supersonic particles do not necessarily have to accommodate the majority of the deformation but the substrate, now thermally softened, can more readily deform and assist in increasing the adhesion of the particles upon impact \cite{RN173}. Though one could heat a substrate via electrical resistance or some warming back plate, this would provide a continuous annealing treatment over the entirety of the substrate and coupling the cumulative time of the deposit could result in deleterious microstructure evolution. Furthermore, in a repair application, it may not be possible to anneal the base material. By LACS, the necessary heat is focused and directly applied to the deposit surface with controlled time at temperature and spatial positions. LACS has successfully deposited and improved the deposition of Ti \cite{RN173}, Cu-based cermets \cite{RN3366}, stellite \cite{RN191,RN135,RN780,RN1406,RN1392,RN1390}, tungsten \cite{RN156}, tungsten carbide \cite{RN2245}, and even ODS materials \cite{RN383}. 



	To date, several of these LACS studies are for single layer deposits. Ultimately, the continual deposition of the material on subsequent layers of the same material is required for larger scale structures and repairs. Though particle-to-particle deposition has and will occur in a single deposition pass, the evolution of the layered pass on another layer can create distinct interfacial bonding features \cite{RN561,RN147}. As described above in CS deposition, bonding between the particles and the substrate is achieved through severe plastic deformation of the particle with a minor amount of deformation of the substrate. In a layered deposition scenario, the underlining layer of the former deposit becomes the substrate for the subsequent layer. Thus, several interesting phenomena can now occur with LACS. These phenomena include thermal softening for the layer that once was $``$too hard to deposit$"$  but now serves as the substrate, the microstructure stability of the layer under the laser, and the environmental reactions (oxidation) of the layer from the laser heat, to name a few. For example, in the prior deposition of a single-pass ODS material, where the input laser energy was studied, the oxide particles as well as the grain size increased with laser energy \cite{RN383}. This increase in deposit energy showed an advantageous increase in deposition efficiency but at an expense of hardness because of these microstructural changes. To date, the deposition of a multi-layered ODS material has yet to be reported. 



	This paper studies the multi-layered LACS of the ODS Fe\textsubscript{91}Ni\textsubscript{8}Zr\textsubscript{1 }(at. $\%$) alloy, which is the same composition from the previous successful deposition from a single pass \cite{RN383}.  We deposited a centimeter square footage structure from six distinct passes yielding a thickness that was a few millimeters. Furthermore, we deposited two runs, one at 650 \celsius{} and 950 \celsius{}. The 650 \celsius{} deposit retains the ferritic state of the ODS alloy while the 950 \celsius{} places the alloy in the austenite phase field during deposition, offering two distinct crystal structures that occurred during deposition. This particular ODS alloy forms its nanoscale oxides from the intrinsic reactions of Zr with oxygen impurities picked up during the ball milling of the constituent metal powders \cite{RN740}. The details of its alloy development are found in references \cite{RN267,RN740,RN966,RN243,RN550,RN1376,RN291,RN476}. Unlike the prior ODS powder development, which was done using a shaking mechanism for ball milling, the powders here have been processed by Zoz ball milling so that a larger volume yield per batch may be achieved. This difference in the milling mechanism results in modifications in imparted energy to the milling process with potential variances in starting crystallite sizes and powder morphology. As of such, this paper will address a detailed characterization of the Zoz milled powders coupled with the multi-layered ODS LACS deposition as this has yet to be reported for this ODS alloy. 





	\section*{Experimental Methods}

		\subsection*{Powder Synthesis} 

			Elemental Fe, Ni, and Zr powders with a mixture composition of Fe\textsubscript{91}Ni\textsubscript{8}Zr\textsubscript{1 }(at. $\%$) were Zoz ball milled in a Simolayer CM08 ball mill (Zoz, Wenden, Germany) in an Ar atmosphere. The mill ran at 400 revolutions per minute for 30 hours while the temperature was kept at -25 \celsius{} for 30 hours. The ball-to-powder ratio was 10:1 using 440C stainless steel balls. During this milling, the Zr reacted with ambient oxygen impurities present in the metal powders creating the nano-scale oxides for the ODS alloy.



		\subsection*{LACS Deposition}



			Substrates for the LACS experiments were low-carbon AISI 1018 steel plates (Grainger, Lake Forest, IL) with dimensions of 75 mm (width) x 300 mm (length) x 12.6 mm (thickness). In order to absorb sufficient laser light, the substrates were polished with P-90 grit that roughened the surface reducing its spectral scattering. The substrates were washed in soapy water and rinsed with isopropyl alcohol prior to deposition.
		
		
		
			CS deposition was performed using a high-pressure cold spray VRC Gen III system controlled with a Viper robotics system (VRC Metal Systems, Box Elder, SD). The sprays used a nominal 75He/25N\textsubscript{2 }(vol. $\%$ ) gas mixture at a gas temperature of 600 \celsius{} and a gas pressure of 4.48 MPa. The multi-layered structure was built using a double-back raster pattern that involved the nozzle moving in the positive straight-line x-direction for 27 mm, then moved in the positive y-direction for 1.5 mm, then reversed in the negative x-direction for 27 mm. The nozzle moved in this manner for 37 passes for a total movements of 54 mm in the y-direction. After the end of the passes, the nozzle immediately turned around and rastered back and forth in the x-direction while stepping 1.5 mm in the negative y-direction. The nozzle moved up in the positive z-direction (away from the substrate) by 0.1 mm after each layer. This deposition patterning resulted in a structure that is 27 mm x 54 mm with a thickness dependent on the amount of material deposited. For all samples, six layers were deposited this manner. The nozzle velocity was 25 mm/s and accelerated at a nominal 1000 mm s\textsuperscript{-2}.
		
		
		
			An off-axis infrared (wavelength, $ \lambda $  = 940 nm) LDM-4000-100 variable power (maximum 4 kW) diode laser (Laserline, Mülheim-Kärlich, Germany) was attached to the CS nozzle system at an angle of 29.1 $ ^{\circ} $  from the spray plume \cite{RN720}. The laser and cold spray nozzle arrangement was such that the laser illuminated an approximate 8 mm diameter spot directly underneath the nozzle. Because the laser and nozzle are attached, the laser moves with the nozzle so that the area underneath the nozzle is always irradiated. A pyrometer was configured in such a way that the temperature of the spot illuminated by the laser could be continuously recorded and placed in a feedback loop with the laser. As the temperature changed, the laser power adjusted to maintain the same temperature. Laser power adjustments were made in 10 ms time-steps allowing for a near-steady surface temperature to be present at all times during deposition. Several temperatures were tested, but only the two highest temperature sprays created a deposit of any mention. The LACS deposits used in this study had a surface temperature of 650 \celsius{} and 950 \celsius{}.



	 \subsection*{Characterization} 



		X-ray diffraction data of the powders were collected using a G-8 GADDS system (Bruker, Billerica, MA) with a CoK$ \alpha $  ($ \lambda $  = 1.789Å) source operated at 40 kV and 35 mA with a wide field-of-view detector allowing a 0.005 $ ^{\circ} $  detectable stepsize. The scan lasted for 90 minutes.
	
	
	
		Mass deposition efficiency from the deposition was calculated by dividing the difference of mass deposited on the substrate by the mass difference in the powder feeder. The LACS deposits were sectioned normal to the longitudinal nozzle movement direction (x-direction). They were mounted, ground using successively finer grits, and finally polished with a 9  $ \mu $m, 3  $ \mu $m, and 1  $ \mu $m diamond paste slurry. For electron back-scattered diffraction (EBSD) surface preparation, the samples underwent an additional vibratory polish in colloidal silica for 4.5 hrs. Hardness measurements were collected using an automatic Vickers hardness tester (Clemex, Quebec, Canada) using 2.94 N (300 gf) held for 10 sec.
	
	
	
		Visible-light micrographs were obtained with an inverted metallurgical microscope (AmScope, Irvine, CA). Scanning electron microscopy (SEM) micrographs of the deposits were imaged using either a FE 7000 scanning electron microscope (JEOL, Peabody, MA), an Apreo SEM (Thermo-Fisher Scientific, Waltham, MA), or a Lyra 3 SEM (Tescan, Brno, Czechia). Energy Dispersion Spectroscopy (EDS) data were collected from an Octane Elite Super Si Drift Detector (EDAX, Draper, UT) with EBSD images collected from an Inca Camera platform (Oxford Instruments, Abingdon, UK) attached to the SEM. EBSD patterns were indexed using the OIM analysis package (EDAX). Body-centered cubic (BCC), face-centered cubic (FCC) and hexagonal close packed (HCP) crystal structures were the searchable phases; no HCP crystal structures were not found in any of the scans. Strain-induced martensite was not detected in the cold spray of the ODS materials, though it is known to be difficult to identify by EBSD \cite{RN3446}. Furthermore, one would not necessarily expect significant martensite since these ODS alloys have a very low carbon contaminate level ($<$ 0.4 at. $\%$ ) \cite{RN267}. Grain dilation was used to clean all EBSD grain reconstructions. 
	
	
	
		 For atom probe tomography (APT), the samples were fabricated through a focused ion beam (FIB) lift-out method \cite{RN347}. A Pt bar that was 1  $ \mu $m thick was deposited onto the surface of the ODS deposit using the gas infiltration system (GIS) on the Quanta Dual-Beam FIB (Thermo-Fischer Scientific). With the protective Pt on the surface, trenches on either side of the Pt was FIB milled at 30 kV and 3 nA creating a wedge of the sample, which was extracted using an in situ micromanipulator (Omniprobe, Oxford Instruments) and mounted with GIS weldment to a series of pre-sharpened flat top posts (Cameca). The sample was then FIB milled into the required needle-like geometry for field evaporation using a sequential reduction of larger to smaller annular rings with ever decreasing milling currents from 200 pA to 100 pA to 50 pA at 30 kV. The tip was milled until it was approximately 100 nm in diameter. Afterwards, the tip was $``$cleaned$"$  at 5 kV and 40 pA to reduce any Ga\textsuperscript{+} surface damage created during milling. The tips were stored under vacuum until they were analyzed using APT.
	
	
	
		 A LEAP 5000 XS (Cameca, Madison, WI) performed the APT. The tips were cooled to 40 K and kept at $<$ 1E-8 Pa. A direct current voltage was applied as well as a pulsing laser at 100 pA. The voltage increased as the tip evaporated to maintain a 1.0 $\%$  detection rate for the number of recorded evaporation events per laser pulses. The pulse rate systematically increased as the voltage increased from 200 kHz to 650 kHz. Collected data was reconstructed and analyzed using IVAS 3.8 (Cameca) and MATLAB (MathWorks, Natick, MA) following oxide measuring procedures found in references \cite{RN1023,RN2626}. 


	\section*{Results and Discussion}



		\subsection*{ODS Powder}

			The post-ball milled powder particle morphology is shown in the SEM images in \ref{fig:ODSLACS1}(a) and (b). From \ref{fig:ODSLACS1}(a), the effective radius was determined by the area fraction from which a cumulative effective volume fraction was calculated and plotted in \ref{fig:ODSLACS1}(c). The median size was determined to have a diameter of 11.6 mm with particle diameters up to approximately 160 mm. The EDS spatial chemical map of the powders is displayed in \ref{fig:ODSLACS1}(d) to (f). In large part, the powders reveal a relative uniform signal from the three initial metals suggesting sufficient homogeneous mixing. The oxide clusters are not observed because of their nanoscale size. The forthcoming APT will characterize this phase. Nevertheless, some Ni-rich regions were noted in these chemical maps, \ref{fig:ODSLACS1}(e), indicating that some inhomogeneity is present. The XRD scan, \ref{fig:ODSLACS2}, did not reveal a Ni phase indicating that any inhomogeneity phase fraction is less than the detection limit of XRD, which is about $<$1 $\%$ .




			\begin{figure}
				\centering
				\includegraphics[width=6.27in]{fig/ODSLACS/Picture1.png}
				\caption[Micrographs and data of the milled Fe\textsubscript{91}Ni\textsubscript{8}Zr\textsubscript{1 }(at. $\%$) powder.]{Micrographs and data of the milled Fe\textsubscript{91}Ni\textsubscript{8}Zr\textsubscript{1 }(at. $\%$) powder. \textbf{(a) }Cross-sectioned BSE images. \textbf{(b)} SEM images of unmounted powder. \textbf{(c) }Cumulative volume fraction as a function of particle diameter. \textbf{(d-f)} EDS elemental maps of cross-sectioned powders including \textbf{(d)} Fe, \textbf{(e)} Ni, and \textbf{(f)} Zr.}
				\label{fig:ODSLACS1}
			\end{figure}




			 The XRD scan reveals the BCC phase. The breadth of the peaks reveal a crystallite size of 277 $ \pm $  88 nm as calculated through Scherrer analysis \cite{RN3429}. Prior XRD analysis of this alloy produced by the SPEX mill, revealed a crystallite size that was approximately one-tenth this size \cite{RN550}. The smaller crystallite size from the SPEX mill is a result of the higher impact energy as compared to the Zoz milling. 



			\begin{figure}
				\centering
				\includegraphics[width=6.27in]{fig/ODSLACS/Picture2.png}
				\caption{XRD of the starting Fe\textsubscript{91}Ni\textsubscript{8}Zr\textsubscript{1 }powder with accompanying indexed planes.}
				\label{fig:ODSLACS2}
			\end{figure}



	 		Despite the differences in crystallite sizes, the Zoz ball milling still creates the nanoscale oxides because of the intrinsic reactivity of Zr. The APT maps, \ref{fig:ODSLACS3}, clearly reveals the nano-scale oxides. \ref{fig:ODSLACS3}(a)i and 3(b)i represents 1 at. $\%$  Zr (confidence $ \sigma $ -1) isoconcentration surface shown by the red-colored surfaces with \ref{fig:ODSLACS3}(a)ii and 3(b)ii identification of the oxide clusters using a maximum separation algorithm. From the isoconcentration surfaces, a proximity histogram is shown, \ref{fig:ODSLACS3}(a)iii and 3(b)iii. The Fe and Ni concentrations decrease closer towards the center of the particles whereas the Zr and O increases closer towards the interior of the particles. The ratio of Zr:O is near 1:3 instead of the nominal 1:2 ratio that is to be expected in ZrO\textsubscript{2}. This discrepancy is the result of extra O reacting with Fe forming Fe\textsubscript{x}O\textsubscript{y }species in addition to zirconia. Trace amounts of C and N were also noted in the APT profiles, indicating additional impurity uptake to the oxides from the powders themselves during milling. Prior APT work from the SPEX mill ODS powders revealed similar contaminants as well as a chromium shell structure around some of the particles \cite{RN267}, which is not observed in these cases. Though both milling processes used stainless steel containers and milling balls, the higher energy from the SPEX mill appears to facilitate the Cr pick-up from the vial as compared to Zoz milling. The average number density from these two APT data sets in \ref{fig:ODSLACS3} is 6.59E23 $ \pm $  1.54E23 m\textsuperscript{3}, with the standard error determined from the two sets. 


			\begin{figure}
				\centering
				\includegraphics[width=6.27in]{fig/ODSLACS/Picture3.png}
				\caption[Two atom probe tomography datasets of ball-milled Fe\textsubscript{91}Ni\textsubscript{8}Zr\textsubscript{1 }powder.]{Two 75-million ion atom probe tomography datasets of ball-milled Fe\textsubscript{91}Ni\textsubscript{8}Zr\textsubscript{1 }powder. \textbf{(a\textsubscript{i} and b\textsubscript{i}) }Fe ion map with Zr isoconcentration surfaces (1 at. $\%$  Zr at a $ \sigma $ -1 confidence). (\textbf{a\textsubscript{ii} and b\textsubscript{ii}) }Maximum separation cluster analysis with \textit{d\textsubscript{max}}\textsubscript{ }\textbf{= }0.60 nm, \textit{Order} = 1, \textit{L} = 0.60 nm and \textit{d\textsubscript{erosion}} = 0.60 nm. \textbf{(a\textsubscript{iii} and b\textsubscript{iii}) }Proximity histograms of the isoconcentration surfaces delineating the matrix and the particle.}
				\label{fig:ODSLACS3}
			\end{figure}




		\subsection*{Laser Assisted Cold Spray Deposits}



			The multi-layered deposits are shown below in \ref{fig:ODSLACS4}. In these images, a clear qualitative difference is seen. While the 650 \celsius{} deposit is arguably more difficult to discern from the substrate, this deposit has noticeable craters and pores that are throughout the entirety of its deposited surface. Even a crack is apparent near the bottom portion of the image of the deposit. In contrast, the 950 \celsius{} deposit reveals a notable build-up on the substrate, both by the distinct color difference with the substrate as well as periodic peaks and valleys from its ribbed surface topography. This ribbed topography is a result of the spray deposit as the nozzle transverses the substrate. There are no apparent cracks or signs of delamination in the 950 \celsius{} deposit. 


			\begin{figure}
				\centering
				\includegraphics[width=6.27in]{fig/ODSLACS/Picture4.png}
				\caption{Photographs of multi-layered LACS deposits of ODS Fe\textsubscript{91}Ni\textsubscript{8}Zr\textsubscript{1 }deposited at surface temperatures of\textbf{ (a) }650 \celsius{} and \textbf{(b)} 950 \celsius{}.}
				\label{fig:ODSLACS4}
			\end{figure}



	 		The mass deposition efficiencies and thicknesses are tabulated in \ref{tab:ODSLACS1} and compared against the single pass deposit reported in Story et al. for the same alloy \cite{RN383}. It is worth to note that the single deposits from reference \cite{RN383} were on 314 stainless steel as compared to the 1018 steel substrate used in this study. This difference will compound any direct comparisons but some general observations can be gleaned. For the 650 \celsius{} deposit, the deposition efficiency is nearly 2X larger for the single pass than the multi-layered deposition. This difference is linked to the surface the deposit must adhere to. Though the 314 stainless steel substrate is harder (Vickers 190) than the 1018 steel substrate (Vickers 130), it is not nearly as hard as the ODS deposit (\ref{tab:ODSLACS1}). As a result, once the ODS deposit places a layer of itself over the 1018 steel substrate, the subsequent ODS layer is now needing this newly deposited ODS layer to exhibit sufficient thermal softening to increase its plasticity to enable a subsequent ODS layer to deposit onto it. Since the multi-layer deposition efficiency is reduced as compared to the single layer deposition at 650 \celsius{}, it is clear that insufficient thermal softening of the ODS layers occurs at this temperature. Since 650 \celsius{} is in the ferritic phase field during the entirety of the deposition run, with limited plasticity as compared to the austenitic phase, this difference in deposition efficiency appears to have a contribution to the crystallographic phase of the deposit substrate during deposition. Comparing the 950 \celsius{}, where the ODS alloy would be austenitic at this temperature, the deposition efficiency is the highest for this multi-layered sample than for any other condition including the single pass deposit on stainless at the same higher temperature, \ref{tab:ODSLACS1}. Furthermore, we note that the single pass 950 \celsius{} deposition efficiency also increased by approximately 20 $\%$  from the single pass 650 \celsius{}, though its increase is not as dramatic as the difference between the multi-layered ODS passes between these two temperatures, which was nearly a 78 $\%$  increase in efficiency. From this comparison it is apparent that the ODS deposit is quite sensitive to the phase state of the iron matrix for accommodating the necessary plasticity for the cold spray deposition. Because of the increased deposition efficiency, the deposit increased considerably in thickness between the two multi-layer deposits as previously in \ref{fig:ODSLACS4}. 



	


\begin{table}[]
	\caption[The relationship between spray deposit conditions, deposition efficiency, and cross section Vickers hardness.]{The relationship between spray deposit conditions, deposition efficiency, and cross section Vickers hardness. The data for the single-layer deposits were retrieved by \cite{RN383}. Note that a specific deposit thickness was not reported in the paper therefore it is not tabulated in here.}
	
\resizebox{\textwidth}{!}{%


	\begin{tabular}{llll}
		\toprule
	{ \textbf{Spray Conditions}} & { \textbf{Deposition Efficiency (\%)}} & { \textbf{Thickness (mm)}} & { \textbf{Cross-section Vickers Hardness}} \\
	\midrule
	{ 650 °C 6 layers}          & { 7.3}                        & { $\sim$0.7}              & { 598 ± 56}                         \\
	{ 650 °C 1 layer}           & { 13.8}                       & { ---}                    & { 333 ± 4}                         \\
	{ 950 °C 6 layers}          & { 32.4}                       & { $\sim$3.1}              & { 293 ± 38}                        \\
	{ 950 °C 1 layer}           & { 17.2}                       & { ---}                    & { 280 ± 6}                        
	\end{tabular}


}

	\label{tab:ODSLACS1}
\end{table}



	Visible-light optical microscopy of the cross-sections normal to the raster direction, \ref{fig:ODSLACS5}, reveals the detailed differences between the LACS deposit between the two deposition temperatures. As noted in \ref{fig:ODSLACS4}(a), where cracks where observed, the cross-section of the 650 \celsius{} sample further confirms their presence as well as the poor adhesion of the deposit. One such delamination region is at the deposit/substrate interface and the second being within the deposit itself and is located approximately halfway between the substrate and top surface of the deposit, \ref{fig:ODSLACS5}(a) – (b). Since the initial observation of the deposit, \ref{fig:ODSLACS4}(a), revealed that the deposit was intact, the delamination is suspected to have occurred as the sample was metallographically prepared for imaging. It is likely that cutting into the material to mount and polish the sample released the residual stresses in the coating that developed upon deposition. In a prior study by the authors on the residual stress of these deposits \cite{RN3459} the surface of the deposit was shown to have compressive residual stress that were approximately -400 MPa. Since the surface of the deposit is in compression, mechanical equilibrium dictates that tensile stresses must reside elsewhere in the deposit with such tensile residual stresses causes the deposit to delaminate. 


			\begin{figure}
				\centering
				\includegraphics[width=6.27in]{fig/ODSLACS/Picture5.png}
				\caption[Visible-light micrographs of cross-sections of LACS of ODS Fe\textsubscript{91}Ni\textsubscript{8}Zr\textsubscript{1 }normal to the nozzle movement direction.]{Visible-light micrographs of cross-sections of LACS of ODS Fe\textsubscript{91}Ni\textsubscript{8}Zr\textsubscript{1 }normal to the nozzle movement direction.\textsubscript{ }\textbf{(a)} LACS deposits at the deposition surface temperature of 650 \celsius{}. \textbf{(b)} A different location laterally of the same sample. \textbf{(c)} Bottom of the deposit including the deposit-substrate interface of a LACS deposit with a surface temperature of 950 \celsius{}. \textbf{(d) }Top of the deposit including the surface.}
				\label{fig:ODSLACS5}
			\end{figure}





	These images are in stark contrast to the cross-section optical micrographs of the 950 \celsius{} deposit, \ref{fig:ODSLACS5}(c)-(d). Since this deposit was sufficiently thick, \ref{tab:ODSLACS1}, the images were divided into two micrographs, one at the substrate interface, \ref{fig:ODSLACS5}(c), and one near the upper portion of the deposit, \ref{fig:ODSLACS5}(d). In either image, which was representative across the entirety of the deposit, no apparent voids or other defects were observed in the consolidated deposit. Unlike the 650 \celsius{} sample, the 950 \celsius{} deposit was retained on the substrate, i.e. did not delaminate. Though it would also have residual stresses, the higher temperature and significantly improved deposition quality indicate that the metallurgical bonding within the deposit and that to the substrate was sufficient to mitigate the prior decoupling of the deposit from the substrate noted at 650 \celsius{}. 



	The dark lines in the deposits in \ref{fig:ODSLACS5} delineate the different layers during the deposition. Though the deposition involved 6 sequential layers, for the 950 \celsius{} deposit, approximately 30 individual layers can be counted from the bottom to top of the deposit. These additional layers are an artifact of how the nozzle shifts 1.5 mm during the raster as it creates each pass. During the spraying process, the powders emerge from the divergent nozzle creating a distribution of powder deposition over the substrate, with the maximum deposit directly under the nozzle. As a result, the peripheral deposits overlay each other with each raster and each subsequent pass creating the observed multiple layering morphology. Because of the differences in deposition thicknesses (deposits directly under the nozzle verses those at the peripheral), there are thicker and thinner deposit regions. These differences in the thick and thin deposited regions are exacerbated by the fact that the cold spray deposition is optimized when the substrate is orthogonal to the gas spray plume but decreases at a greater magnitude as the angle between the substrate and the gas spray plume is increased \cite{RN500}. This deposition anisotropy, particularly for the 950 \celsius{} sample, is confirmed further by the surface topography, \ref{fig:ODSLACS4}(b), which showed the ‘ribbed’ material build up on the surface. 



	 The dark lines that revealed the layering are a result of surface oxidation created by the laser heat. Inspection of the EDS map of the 950 \celsius{} deposit, \ref{fig:ODSLACS6}, reveals the delineated oxygen signal at these interfaces with progressively less oxygen detection within the layered material. As Fe readily oxidizes with increasing temperature, the top deposited surface formed a thin scale. This scale layer is then subsequently ‘buried’ upon the next deposit, which then has its top surface forming the oxide scale and so forth. The formation of these scales could be detrimental to mechanical properties by preventing elemental metallic layer bonding. In future studies, the use of a shielding gas, similarly done in fusion welding, will be explored. 




				\begin{figure}
					\centering
					\includegraphics[width=6.27in]{fig/ODSLACS/Picture6.png}
					\caption[EDS maps of a cross section of LACS deposited Fe\textsubscript{91}Ni\textsubscript{8}Zr\textsubscript{1 }at a substrate temperature of 950 \celsius{}.]{EDS maps of a cross section of LACS deposited Fe\textsubscript{91}Ni\textsubscript{8}Zr\textsubscript{1 }at a substrate temperature of 950 \celsius{}. The maps show the presence of \textbf{(a)} Fe and \textbf{(b)} O.}
					\label{fig:ODSLACS6}
				\end{figure}






	To further understand the LACS deposit, we examine the morphological shape of the powder upon its impact in the deposit. In a typical cold spray deposit, the microstructure reveals severe plastic deformation within the deposited powders \cite{RN486}; however, in LACS, the extent of this deformation of the deposited powder particle is not as readily apparent, as seen in \ref{fig:ODSLACS7}. This difference is a result of the substrate now accommodating a significant portion of the deformation energy upon particle impact because of its thermal softening created by the local heating of the laser. This thermal softening mechanism from heat transfer has been clearly reported in a prior LACS study of Ti \cite{RN173} and will be expanded upon in the subsequent discussion. 


				\begin{figure}
					\centering
					\includegraphics[width=6.27in]{fig/ODSLACS/Picture7.png}
					\caption[BSE micrographs of the cross sections of LACS deposits of ODS Fe\textsubscript{91}Ni\textsubscript{8}Zr\textsubscript{1 }at \textbf{(a)} 650 \celsius{} and \textbf{(b) }950 \celsius{} showing the top (surface), middle, and bottom (deposit/substrate interface) of the deposit.]{BSE micrographs of the cross sections of LACS deposits of ODS Fe\textsubscript{91}Ni\textsubscript{8}Zr\textsubscript{1 }at \textbf{(a)} 650 \celsius{} and \textbf{(b) }950 \celsius{} showing the top (surface), middle, and bottom (deposit/substrate interface) of the deposit. For special interest, \textbf{(c)} an image of the interface between first and second layer of the 950 \celsius{} LACS deposit is provided in the third column.}
					\label{fig:ODSLACS7}
				\end{figure}



	In \ref{fig:ODSLACS7}, we do not easily discern any of the non-symmetric, polygonal shapes of the original particles, \ref{fig:ODSLACS1}. While large spherical particles are conclusively connected to a lower critical velocity (velocity needed to achieve deposition) than small spherical particles \cite{RN3458}, it is not entirely certain exactly how non-spherical particles behave during cold spray. Some studies have suggested that irregular-shaped particles improve the deposition efficiency more than spherical particles because the irregular-shaped particles ‘catch the gas stream’ more effectively \cite{RN511,RN3428}. If the irregular shaped particles were indeed the result of the increased deposition efficiency reported here, then a larger fraction of irregular-shaped particles would present in the micrographs of the deposits, which is not necessarily the case. Rather, the majority of the particles in the micrographs, \ref{fig:ODSLACS7}, appear to have a flattened spherical shape. Furthermore, the dependence of the deposition efficiency with temperature further supports that the laser, and not the powder morphology, plays a more significant impact on the deposition efficiency outcome. 



	 The oxide scales between the deposited layers is clearly seen in the wider field of view of the entirety of the deposits in \ref{fig:ODSLACS7}, with the oxide contrast most pronounced in the 950 \celsius{} deposit. The higher temperature caused the initial 1018 substrate to even oxidize slightly between it and the first deposit, evident by the dark contrast seen in \ref{fig:ODSLACS7}(b)-bottom. One imaging advantage of these oxide scales is its ability to delineate the powder particles in the deposit. This is shown in the magnified image of the first and second layer deposit at 950 \celsius{} in \ref{fig:ODSLACS7}(c). The particles are relatively round, as previously discussed above. The lack of a ‘flattened’ particle morphology, as shown by this particle, also confirms the lower deformation it received upon impact as well as the thermal softening it underwent as it served as the substrate for the subsequent deposits. 



	Of all the regions of these deposits, the top most layers reveal the clearest powder-in-layer morphologies. In the 950 \celsius{} condition, the particles’ lower portions were rather round while their top portions appear flat in shape as indicated by the arrows in \ref{fig:ODSLACS7}(b) The ability for the lower portion to retain a round shape further indicates that the substrate, i.e. the previous deposited ODS layer, was deforming more than the impacted particle itself. If that was not the case, this portion of the particle would be compressed and flat. 



	The ‘flattened’ powder shapes are more evident in the 650 \celsius{} deposit, as indicated by the arrows in \ref{fig:ODSLACS7}(a). The flattening of the particle is indicative of the powder itself undergoing the deformation induced by the plastic strain shape change. Unlike the higher temperature deposit, the substrate layers were insufficiently thermally softened and/or not the correct crystal structure type, i.e. ferritic, to accommodate the deformation of the arriving particle. This transferred the energy back into the particle resulting in its morphological change. 



	Finally, using the cross-sectional images of \ref{fig:ODSLACS7} (as well as other micrographs not shown), the surface roughness between the two different deposition temperatures was quantified. The 650 \celsius{} condition has a root mean square surface roughness of 70 $ \pm $  61  $ \mu $m and the 950 \celsius{} condition having a root mean square roughness of 48 $ \pm $  28  $ \mu $m. This difference in the mean size as well as its associated standard error is created by a shot peening or tamping effect of the cold spray powder. At the lower temperature, there was insufficient velocity to adhere the particles to the substrate. Consequently the powder particles impacted and then rebounded off the surface, evident by a lower deposition efficiency value, creating large variations over the surface. In contrast, the improved adhesion of the power particles at the elevated temperature reduced this roughening effect as more particles were able to be retained on the surface keeping it relatively smoother. 



	With the increase in deposition temperature, one would expect a variation in the grain sizes between the two deposits. \ref{fig:ODSLACS8}(a) – (b) are a series of EBSD scans between the deposited temperatures as well as differences between the top and bottom portions within a single deposit. As would be expected, the higher temperature resulted in a larger grain size distribution as compared to the lower deposited temperature. This is quantitatively shown in the cumulative grain size distribution plot of \ref{fig:ODSLACS8}(c). For the 950 \celsius{} deposit, the grain size distributions between the top and lower regions are relatively similar, though a modest reduction of grain sizes are noted at the substrate interface and will be expanded upon further for the 650 \celsius{} deposit, where this difference is much more pronounced. The higher 950 \celsius{} temperature appear sufficiently high to yield a rather uniform and consistent heating through the entirety of the deposit. 



	What was particularly surprising was the difference in grain sizes between the top and bottom portions for the 650 \celsius{} deposit. The grains at the substrate appeared refined and equiaxed. In contrast, the grains near the surface were larger and even revealed the onset of abnormal grain growth evident by the presence of some larger grains. Abnormal grain growth has been reported in this class of Fe-Ni-Zr alloys \cite{RN740}. Though the explanation of this grain size difference at 650 \celsius{} is not quite fully understood, it is suspected to be dynamic recrystallization. The continuous penning and tampering effect of the particles striking and rebounding off the 650 \celsius{} deposit, which would be pronounce by its low deposition efficiency, coupled with the modest temperature and a relatively thinner deposit thickness, enabled sufficient deformation into the layers near the substrate. This would create a condition of higher deformation (creation of plastic strain in the grains) coupled with a modest temperature (heat) to initiate the dynamic recrystallization. Li et al. have suggested that longer depositions, i.e. more deposition layers, generates more of a peening and tamping effect, with such an effect having a higher contribution to layers closer to the substrate \cite{RN171,RN679}. 




			\begin{figure}
				\centering
				\includegraphics[width=6.27in]{fig/ODSLACS/Picture8.png}
				\caption[BSE and EBSD inverse pole figures of cross sections of LACS deposited Fe\textsubscript{91}Ni\textsubscript{8}Zr\textsubscript{1 }at substrate temperatures of \textbf{(a)} 650 \celsius{} and \textbf{(b)} 950 \celsius{}.]{BSE and EBSD inverse pole figures of cross sections of LACS deposited Fe\textsubscript{91}Ni\textsubscript{8}Zr\textsubscript{1 }at substrate temperatures of \textbf{(a)} 650 \celsius{} and \textbf{(b)} 950 \celsius{}. The BSE micrographs show the approximate location of where the EBSD images were taken. \textbf{(c) }Cumulative area fraction as a function of grain diameter of only the deposits of both 650 \celsius{} and 950 \celsius{} at both the bottom and the top.}
				\label{fig:ODSLACS8}
			\end{figure}



	 Finally, we evaluated the hardness of the two deposits, with the Vickers hardness values tabulated in \ref{tab:ODSLACS1}. The hardness values were taken in the cross-section. The 650 \celsius{} deposit was nearly 2X harder than the 950 \celsius{} deposit. This difference is contributed to three mechanisms. The first being the grain size differences between the two depositions, \ref{fig:ODSLACS8}(c), with the higher hardness being associated with the smaller grain size deposit for the 650 \celsius{} deposit. This grain size dependence is further highlighted by the scatter plot of hardness shown in \ref{fig:ODSLACS9}. For the 650 \celsius{}, which showed a bimodal grain size distribution, \ref{fig:ODSLACS8}(a) and (c), the smaller refined grains at the substrate interface have a higher clustering of hardness than those of the coarser grains that were near the surface. For the 950 \celsius{} deposit, the hardness values are also clustered in this scatter plot, but with less separation than found in the 650 \celsius{} deposit because of their similar sizes, \ref{fig:ODSLACS8}(b) and (c). 




			\begin{figure}
				\centering
				\includegraphics[width=6.27in]{fig/ODSLACS/Picture9.png}
				\caption[Vickers hardness scatterplot of the cross sections of the LACS deposits taken across the sample.]{Vickers hardness scatterplot of the cross sections of the LACS deposits taken across the sample. Binomial distribution is highlighted by the circles identifying the clusters.}
				\label{fig:ODSLACS9}
			\end{figure}


	The second contributing factor are the pinning nanoscale oxides. Besides the temperature increasing the grain size, the heat would also coarsen the nanoscale oxide particles. \ref{fig:ODSLACS10} is a plot of the oxide particles’ equivalent spherical radius between the pre-sprayed powders, \ref{fig:ODSLACS3}, and the atom probe data set reported in \cite{RN383} for the single pass deposit at 950 \celsius{}. With increasing temperature, the spread in oxide particle sizes is apparent, \ref{fig:ODSLACS10}(a). The median volume equivalent radius of the pre-deposited particles is 1.3 nm whereas the LACS deposit is approximately 2.4 nm, more than 1 nm larger. Though the small volume particles are still plentiful in the LACS deposit (1.85E23 $ \pm $  1.11E23 m\textsuperscript{-3} \cite{RN383} versus 6.59E23 $ \pm $  1.54E23 m\textsuperscript{3 }(powders)), it is clear that some particle coarsening has occurred. Since an Orowan-based strengthening mechanism is a function of the average spacing of these particles \cite{RN119,RN3414}, the relative increase of inter-particle spacing is congruent with a decrease in a number density and an increase in the particle volume. As the oxides space further apart, they contribute less to this high strength mechanism. 



	The third mechanism is the extra peening and tamping that occurred at 650 \celsius{} as compared to the 950 \celsius{} deposit. This effect is most evident when one compares the single deposit hardness (Vickers 333 $ \pm $  4) to the six-layer deposit (Vickers 598 $ \pm $  56). With increased deposition layers, there was an increase in peening and tamping which also would contribute to an increase in the hardness. The collective findings reveal the intrinsic influences of LACS processing on the microstructure and its corresponding impact on the mechanical properties, with ample areas for further optimization.



			\begin{figure}
				\centering
				\includegraphics[width=6.27in]{fig/ODSLACS/Picture10.png}
				\caption[Atom probe tomography data showing nano-scale oxides of starting powder and single-layer LACS deposits at 950 \celsius{} \cite{RN383}.]{Atom probe tomography data showing nano-scale oxides of starting powder and single-layer LACS deposits at 950 \celsius{} \cite{RN383}. \textbf{(a) }Data frequency histogram combined with a box and whisker plot showing the statistical prevalence of particle sizes represented by a volume equivalent radius. \textbf{(b)} Histogram represented by the relative frequency of the inter-particle spacings between the different nano-scale oxides.}
				\label{fig:ODSLACS10}
			\end{figure}




	\section*{Conclusion}



		 ODS Fe\textsubscript{91}Ni\textsubscript{8}Zr\textsubscript{1 }structures were deposited through a multi-pass, multi-layered laser assisted cold spray (LACS) process at temperatures of 650 \celsius{} and 950 \celsius{}. Deposition efficiency increases with increased surface temperature, from $ \sim $  7$\%$  to 32$\%$ . This increase is associated with the thermal softening of the deposited layers that act as a substrate for the sequential deposition of the next layer. In addition, the ferritic to austenitic phase change that would occur at these temperatures enabled the accommodating deformation. The 650 \celsius{} deposit revealed more flaws in the structure, including cracking and delamination. In contrast, the 950 \celsius{} deposit was consolidated with no evident microstructural defects. With the application of the heat did yield oxidation and the formation of scales between the layered deposits, with this being more evident with increasing temperature. The increase in temperature also promoted grain growth between the two deposits and a subsequent difference in Vickers hardness, with the 650 \celsius{} being 598 $ \pm $  56 and 950 \celsius{} being 293 $ \pm $  38. Interestingly, the 650 \celsius{} deposit revealed a marked difference in grain sizes between those at the substrate and those near the top of the deposit. A much more modest change, but similar trend was seen for the 950 \celsius{} deposit. Near the substrate interface, the 650 \celsius{} grains were equiaxed and refined. The lower deposition efficiency created a more dramatic peening and tamping effect, which has a greater impact for layers near the substrate. Coupling this with the heat enabled these grains to undergo dynamic recrystallization. 
