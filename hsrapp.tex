\chapter{ATOM PROBE MASS-TO-CHARGE RATIO SPECTRUM OF ODS MATERIAL ECAE PROCESSED AT 1000 \celsius.}




\ref{fig:HSRA1} shows the atom probe spectrum of the ODS alloy ECAE processed at 1000 \celsius{} found in Chapter 3 and Chapter 5. In order to deduce the composition, we need to assign specific elements to each peak.  For example, between 33-39 Da, Fe, Ni, and Zr all have multiple isotopes that will contribute to the identified complexes within this range, with several of these complexes overlapping. For clarity, we have only shown the final assigned peak identification. When species overlap was found, the assignment for the peak was then determined by the enthalpy of formation between the overlapping complexes. Considering that zirconium carbide, nitride, or oxide has a greater enthalpy of formation than that of either equivalent iron or nickel carbide, nitride, or oxides \cite{RN3452,RN413,RN557,RN3453,RN3454,}, the peak would then be assigned to the zirconium based complex. Though some fraction of the other complex may be present, it was not included because there is no means to ascertain the amount of that contribution to the total peak. As noted in the main text of the paper, the use of laser-assisted field evaporation can change the relative ratio of the charge states making it difficult, if not impossible, to de-convolute the contribution of each species within the overlapping charge-state peak locations. But, based on the close agreement between the starting composition and the overall composition collected from the APT mass spectrum, this assumption we have applied here appears reasonable.

\begin{figure}
	\centering
	\includegraphics[width=6.27in]{fig/HSR/FigureA1.png}
	\caption[Mass spectrum from the 1000 \celsius{} ECAE processed sample.]{Mass spectrum from the 1000 \celsius{} ECAE processed sample. The peaks assigned match what was used for the computational analysis.}
	\label{fig:HSRA1}
\end{figure}