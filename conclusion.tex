
\section*{Chapter Summary}

	In this dissertation, laser assisted cold spray deposition of the ferritic/martensitic alloys including 4340 steel and oxide dispersion strengthened Fe\textsubscript{91}Ni\textsubscript{8}Zr\textsubscript{1} were studied. In conjunction with this study, high-strain rates of ODS material, residual stress of LACS deposits as well as an atom probe study of nano-scale oxides in the material is given. As gas temperature and pressure increases, deposition efficiency increased. As surface temperature increased, deposition efficiency increased. The change in laser power has clear effects on the microstructure of both ODS Fe\textsubscript{91}Ni\textsubscript{8}Zr\textsubscript{1} and 4340 steel. These effects were laid out in detail in each individual chapter. A high-strain rate study of ODS Fe\textsubscript{91}Ni\textsubscript{8}Zr\textsubscript{1} as well as surface residual stresses of LACS deposits were studied.
	
	Quasi-state and dynamic mechanical testing showed the importance of both nanoscale oxides and interstitial solutes in the plasticity of the ODS Fe\textsubscript{91}Ni\textsubscript{8}Zr\textsubscript{1} materials Chapter 3. A Hopkinson bar test\   covered the microstructure and dynamic strain aging of equal channel angular extruded ODS Fe\textsubscript{91}Ni\textsubscript{8}Zr\textsubscript{1}. After yielding, the ODS material experiences a load drop meaning that the stress required to enact more strain on the material was less after yielding than before yielding. This was first hypothesized that particles were shearing during yielding causing the drop in the load, however, atom probe tomography revealed small amounts of nitrogen and carbon in the material. While commercially pure powders were used, the carbon and nitrogen were either present in the parent material or introduced during the ball milling process. These small amounts of interstitial content created dynamic strain aging. This was proven by quasi-static tensile testing. The quasi-static tests also revealed a load drop after yielding. After yielding, but before failure, the material was relaxed and then tested again. This yield drop was caused by the interstitial solutes pinning the dislocations. After the material loaded for the second time, the load drop was no longer observable as the dislocations were no longer pinned by the solutes. After heat-treating the yielded sample to 200 \celsius{}, the nitrogen and carbon could then diffuse to the dislocations and pin them. A subsequent test revealed that the load drop was once again present. It was found that even though the nano-oxides contribute to both the yield strength and ultimate strength of the material, the interstitials still play a role in the plastic behavior of the material. In this case, the interstitials’ effect on plastic behavior is in the form of dynamic strain aging.
	
	Multi-layer deposits produced by LACS of ODS Fe\textsubscript{91}Ni\textsubscript{8}Zr\textsubscript{1} showed microscale and nanoscale evolution as a function of cumulative heat input Chapter 4. This work expanded on the research done by Story et al. where the author also contributed to this work \cite{RN383}. ODS Fe\textsubscript{91}Ni\textsubscript{8}Zr\textsubscript{1} was LACS deposited in a multi-pass, multi-layered nozzle movement that created a single consolidated structure. The characterization of the nano-scale oxides of the powder and the LACS material showed the grain sizes were larger at six layers than at one layer. The same irregular grain growth found by Kotan et al. by annealing ODS Fe\textsubscript{91}Ni\textsubscript{8}Zr\textsubscript{1} was also be found in LACS deposits \cite{RN740}. Grain sizes near the deposit-substrate interface are notablt smaller than near the top of the deposit suggesting recrystallization through tamping.
	
	This dissertation made the first measurements on residual stresses in LACS-deposited materials and observed compressive stresses on the surfaces of ODS Fe\textsubscript{91}Ni\textsubscript{8}Zr\textsubscript{1} deposits, Chapter 6, and gave a preliminary look at residual stresses of LACS-deposited Fe\textsubscript{91}Ni\textsubscript{8}Zr\textsubscript{1}. The residual stress was measured through x-ray diffraction as calculated by the sin\textsuperscript{2}($ \psi$) d-spacing method. Laser irradiation without cold spray on the substrate surface shows a clear tensile residual stress gradient. In the case of the LACS deposits, it has been found that on the surface, the compressive-causing peening from the cold spray overcame the thermal gradients caused by the laser. With high compressive residual stresses on the surface, it is likely that high tensile residual stresses will be found closer to the deposit/substrate interface. 
	
	This dissertation challenged the commonly-used practice of properly quantifying segregration from isoconcentration surface data measured by atom probe in Chapter 5 and provided an exploration into using the technique. The same dataset from Chapter 5 was explored using isoconcentration surfaces and proximity histograms. Isoconcentration surfaces found in the proprietary software IVAS analyzes the concentrations in individual voxels and then creates a surface spanning the voxels with the same concentration. The parameters used to determine isoconcentration surfaces are usually determined by proximity histograms, but averaging out all the proximity histograms used to measure isoconcentration surfaces does not reflect the reality of clustering. Different techniques to determine isoconcentration parameters were tried including randomizing the data to determine the maximum values that causes false clustering identification through randomizing. This proved ineffective. The only way to properly use isoconcentration surfaces of low-concentration high-partitioning solute is to inspect each particle individually. Otherwise, data obtained from isoconcentration surfaces should be a qualitative analysis instead of a quantitative analysis.
	
	This dissertation made the first successful cold spray and LACS depositions on AISI 4340 steel (Chapter 7). This steel was CS and LACS deposited at different gas temperatures and pressures as well as different surface temperatures caused by laser irradiation. The mechanism of deformation is observed through electron microscopy. The top of the substrate plastically deformed, and the bottoms of the particles remained intact. This is indicative of the softening effect of the laser on the substrate. Grain size grew with increasing surface temperature. Oxidation between particle interfaces increased with increasing temperature and at locations closer to the substrate. Increased time and temperature increased oxidation. Cross-sectional hardnesses of LACS deposits at a temperature (500 \celsius{}) below the austenitizing temperatures were lower than CS deposits but recovered at temperatures above the austenitizing temperature when martensite (observed through back-scattered electron microscopy) was formed through the rapid cooling from the higher temperatures.
	
\section*{Future Work}
	\subsection*{Improvement on LACS Deposition and a Need for Standardized Research}	
		With so many knobs to turn in both the cold spray processing and in the materials system, it is difficult to create a systematic and comparable approach to quantify the real influence of the laser on cold spray. The laser can easily produce noticeable differences in a specific material system that is cold spray deposited using specific processing conditions but comparing the laser effects on cold spray of different materials is difficult. Most material systems will deposit given enough velocity delivered by the gas pressure and temperature. The effect of the laser on 4340 and ODS materials appear to be similar. In contrast, cold spray systems have improved since 2009 when Bray et al. published their work on LACS of Ti. Ti and Ti alloys can now be cold sprayed without a laser with much better deposition efficiencies and microstructures than what Bray et al. suggests is achievable without a laser \cite{RN173}. \ref{fig:depcomp}  describes the problem of attempting to compare different surface temperatures against different LACS experiments. The Ti's deposition efficiency increased several fold, while the Stellite-6 experiment was only conducted at temperatures near the melting point of the material. If we are to properly compare the efficacy of the laser in cold spray, a more systematic approach to the exact effects of the laser on cold spray deposition should now be considered. We are nearing the point where there is an abundance of preliminary research studies on LACS, so that we may begin to work on understanding the more fundamental science on the effects of laser irradiation during severe plastic deformation. This section in no particular order are problems that were faced during LACS research with potential solutions that could improve both deposition and repeatability.
	
		\begin{figure}
			\centering
			\includegraphics[width=6.27in]{fig/conclusion/depcomp.png}
			\caption[Relative deposition efficiency of laser assisted cold spray deposition as a function of homologous temperature (T/T\textsubscript{m}).]{Relative deposition efficiency of laser assisted cold spray deposition as a function of homologous temperature (T/T\textsubscript{m}). Graph adopted from Story \cite{RN720} with data obtained from \cite{RN383,RN173,RN135,RN3366}.}
			\label{fig:depcomp}
		\end{figure}
		
		Using different material substrates and LACS onto already LACS deposited samples could yield interesting insights into the differences of cold spray deposit and other processing methods. All the substrates that were used in this dissertation were the same, AISI 1018. A low-carbon material was chosen for the substrate because of both its low cost and its lower hardness in relation to both 4340 and ODS Fe\textsubscript{91}Ni\textsubscript{8}Zr\textsubscript{1}. With the multiple layers of deposition, Chapter 4, we see a glimpse into how the cold spray deposition onto like- substrates of the material would perform. Comparing deposition of these materials onto substrates produced by normal sintering methods against depositing onto substrates produced by cold spray would be interesting. 
		
		
		A second thing to consider researching is the sample substrate surface. For example, to maintain chemical purity, the samples were sanded and washed, but they were not grit blasted. Several research groups have found it useful to grit blast the substrates before cold spray depositing \cite{RN2289,RN1137}. The practice may introduce unwanted contaminants that could bad for adhesion strength and general mechanical properties. Surface roughness also greatly influences laser irradiation absorption. Bergström shows the influence of different magnitudes and shapes of surface roughness and their overall absorption by the laser light \cite{RN1384}. A study linking roughness and surface treatments to not just cold spray, but also coupling the effects of the roughness with cold spray and laser absorption would be interesting.  
	
	
		The influence of other raster patterns during cold spray would be interesting to explore. All of these samples were created through a simple, double-back raster system in which the nozzle moved back and forth on the x-y plane. At the end of each layer, the nozzle would reverse directions and come back over itself. Instead of reversing directions at the end of a layer, the nozzle could reset at the beginning of the deposit. Also, diagonal directions where each pass slightly increases in length could prove interesting. The ability to control heat input is almost always required for metal alloy additive manufacturing. One such method for heat input movement (in the case of LACS, nozzle and laser movement) could be through a Hilbert pattern, where the geometry makes capital ‘T’ type movements that then moves around in a symmetric pattern. The same area is covered, but the long rows of heat input is avoided. This pattern is used to keep the heat concentrated at a specific position for a longer amount of time but does not revisit the spot thus reducing thermal fatigue. Users of other additive manufacturing techniques are beginning to adopt this patterning system into their systems \cite{RN3439}. A diagram of Hilbert pattern movement is found in \ref{fig:hilbert}.\  This is just one of the possible raster patterns that could result in different deposition qualities and yield different mechanical properties of the material. In a case of the Hilbert pattern, a more in-plane isotropic mechanical behavior would be expected compared some sort of rastering deposition. 
		
		\begin{figure}
			\centering
			\includegraphics[width=6.27in]{fig/conclusion/hilbert.png}
			\caption{\textbf{(a)} Schematic of double-back rastering used for spraying in this dissertation. \textbf{(b)} Diagram of a Hilbert pattern.}
			\label{fig:hilbert}
		\end{figure}	
			
		Single-color pyrometry was used for this cold spray system. Powder plume obscuring part of the substrate, the roughness of the substrate caused by sanding, and the changing roughness due to new materials being constantly laid down can change the temperatures that the pyrometer is reading. For this dissertation’s LACS deposits, emissivity values were listed at 1.0, assuming a near blackbody of the material. This might be close for rough low-carbon steel, but stainless steel, aluminum, and other metallic substrates do not resemble a blackbody at all. Thus, finding correct emissivity values will become increasingly important for proper laser heating. Most of these situations effecting a single-color pyrometer can be mitigated by a two-color pyrometer that focuses on a ratio of light intensity \cite{RN3440}. So long as both wavelengths of light measured are affected the same by the conditions, the surface temperature will read more accurately.
			
			
		Ball-milling powder is a violent process that can change the shape and morphology of the powders. This leads to a more skewed distribution of powder sizes that is difficult to sieve. Through some combination of irregular powder shape and size, nozzles tend to foul and clog more often than with spherical powders. This dissertation is not the first report of irregularly sized particles clogging the nozzle \cite{RN173, RN451}. Qualitatively, LACS adversely affect the nozzle as LACS runs performed at higher temperatures have a greater tendency to clog. Al powder particles clog long metal nozzles leading to the Al cold spray industry to use short (120 mm) polybenzimidazole (PBI) nozzles \cite{RN1890}. The clogging nature of irregular steel particles is not completely researched. If more ball-milled material is to be cold sprayed, understanding the nature of the problem will help lead to solutions. Unfortunately, the cost of a typical WC-Co nozzle is at a cost such that research on this topic has so far been prohibitive. Exhaustive studies on nozzle clogging are not found, but water-cooled nozzles have been used to reduce the amount of clogging in a nozzle \cite{RN3010,RN182,RN3442,RN3443}. Particularly with the extra heat generated by the laser that could possibly be absorbed by the nozzle, water cooling cold spray nozzles could be a way to prevent clogging during LACS deposition and should be explored in the future.
			
			
		Cold spray of ODS material at continuing high gas pressures and temperatures was attempted with a shorter WC-Co nozzle that was 120 mm in length compared to 200 mm in length. \ref{fig:csfail} shows a single layer pass with a surface temperature of 750 \celsius{} at the same gas pressure and temperature conditions as with the longer 200 mm nozzle used in the rest of the dissertation. As can be seen, even at such a high surface temperature caused by the laser irradiation (and a soft 1018 steel substrate is still being used), there is practically no deposition. The only deposition that is to be found is few fortunate particles that have somehow embedded themselves into the substrate but does not seem to create the metallurgical bond. The top of the surface is aggressively shot peened suggesting that even with the laser, a critical velocity is still required to achieve deposition despite the differences in the deposition mechanism.	
		
		
		As was found in both 4340 steel and ODS Fe\textsubscript{91}Ni\textsubscript{8}Zr\textsubscript{1}, an increase in temperature results in oxidation at the layer interfaces. Even though inert He gas is used to blow the powders directly onto the surface, the residual temperatures, or the elevated temperatures caused by depositing a second layer is high enough to oxidize the layer from the ambient conditions around it. In many welding practices, a shielding gas is introduced to the welding point so that the melted material does not oxidize before it cools \cite{RN3361}. For LACS of ferrous materials to be useful in the future, it is recommended that a setup using a shielding gas be used for LACS. An additional challenge to LACS oxidation compared to welding is that welds are typically done in straight lines and not rastered. LACS in an additive manufacturing or additive repair role will heat and reheat the same locations and adjacent locations several times throughout a deposit. In order to avoid the oxidation, LACS at lower temperatures should be used. Otherwise, some inert gas draft mechanism or glovebox will need to be employed to keep the oxidation to a minimum.
	
		\begin{figure}
			\centering
			\includegraphics[width=6.27in]{fig/conclusion/csfail.png}
			\caption[Back-scattered electron micrographs of a cross section of a single-pass LACS deposit of ODS Fe\textsubscript{91}Ni\textsubscript{8}Zr\textsubscript{1} using a WC-Co nozzle with a 120 mm length (instead of the normal 200 mm length).]{Back-scattered electron micrographs of a cross section of a single-pass LACS deposit of ODS Fe\textsubscript{91}Ni\textsubscript{8}Zr\textsubscript{1} using a WC-Co nozzle with a 120 mm length (instead of the normal 200 mm length). Surface temperatures were \textbf{(a)} 500 \celsius{} and \textbf{(b)} 750 \celsius{}.}
			\label{fig:csfail}
		\end{figure}
			
	\subsection*{Understanding Residual Stresses of LACS}
		The residual stresses that were studied and reported in this dissertation gives a preliminary study of residual stress generation of laser assisted cold spray. This study was done using x-ray diffraction that only penetrates at most the top 40 $ \mu$m of the surface. Future studies will need to look at more fundamental causes of residual stresses by looking at single stripes and also look at residual stresses as a function of depth of the deposit.
		
		Even though the LACS runs with the 120 mm length nozzle has poor deposition, their geometry was such that accurate residual stress measurements could be made. \ref{fig:otherrs}a shows the residual stress measurements of single layer LACS deposits (or more accurately, shot peening with \textit{in situ} laser heating). Compared to just the heating with the laser, it was found that although the laser still created a thermal gradient and produced tensile stresses (particularly in areas surrounding the affected regions), the shot-peening effect of the cold spray reduced the amounts of tensile stresses along its tracks. \ref{fig:otherrs}b and \ref{fig:otherrs}c shows the steady-state response of the laser heat treatment and LACS. \ref{fig:otherrs}b is residual stress measurements of the 950 \celsius{} laser-only irradiation at three different locations on the laser track: 10 mm, 50 mm, and 90 mm away from the starting location of the irradiation. True to the pyrometer and power output readings, the laser reaches a steady-state position very quickly and the measured residual stress measurements shows practically no difference within the error. At the single pass LACS at 950 \celsius{}, the residual stresses are slightly lower in magnitude further away from the starting location than at the ending spot. It is uncertain at this time the cause of this and further studies should be conducted.
	
		In addition to steady-state LACS of residual stresses, deeper residual stress techniques would be useful in understanding how the residual stress is distributed throughout a LACS deposited material. This dissertation only explores residual stress measurements through x-ray diffraction. Other methods, particularly neutron diffraction will be important to clarify both the nature of the surface residual stresses found through the x-ray diffraction and how the responding stresses to maintain mechanical equilibrium is distributed in the deposit and substrate \cite{RN1400}. 
		
		\begin{figure}
			\centering
			\includegraphics[width=6.27in]{fig/conclusion/otherrs.png}
			\caption[Residual stress measurements of ODS single-layer deposit/shot peening of LACS attempted with a 120 mm WC-Co nozzle.]{Residual stress measurements of \textbf{(a)}\textit{ }ODS single-layer deposit/shot peening of LACS attempted with a 120 mm WC-Co nozzle. \textbf{(b) }Residual stress measurements of laser-only single pass treatments. The distance measurements listed (10 mm, 50 mm, 90 mm) are in relation to the starting location of the laser and nozzle. \textbf{(c) }Residual stress measurements of single pass LACS of ODS at 950 \celsius{} at measurements listed in relation to the starting location.}
			\label{fig:otherrs}
		\end{figure}
		
	\subsection*{Mechanical Properties of LACS Deposits}
		In this dissertation, the most mechanical behavior collected from LACS deposits were cross-sectional hardness tests and surface residual stresses. The next step to understand LACS processing of materials is to machine mechanical testing coupons and understand how the LACS deposits behave mechanically. This can be done through standardized tensile tests. Also, since the Fe\textsubscript{91}Ni\textsubscript{8}Zr\textsubscript{1} ODS material has a particular application in ballistic resistance, shear-punch tests \cite{RN266,RN303} will help understand mechanical behavior. Both as a function of surface temperature, and different raster geometries may be employed to further understand how they contribute to the mechanical properties of the material. 
	
	\subsection*{Atom Probe Data Quality: Mass Spectrum Resolution}
		As is mentioned in \ref{chap:Methods} and shown in Figure A.1 found in appendix section A, attempts to range the atom probe mass-to-charge ratio spectrum has some fundamental challenges. Laser-assisted atom probe tomography allows for nonconductive materials to field evaporate and be measured. This allows for ODS materials to be seen in the atom probe where voltage pulsing would likely cause premature fracturing. A major concern and potential cause for error found in laser-assisted local electrode atom probe is that the laser is known to cause peak broadening that is weighted towards higher Daltons that causes a tail on the ends of the peaks. These so-called $``$thermal tails$"$  are thought mostly to occur by thermal residual effects that cause delayed evaporation leading to a slightly higher mass recording than the true mass of the ion. If they are large enough, the tails of ions of higher concentration in a material can smother other peaks represented by ions with lower concentrations. Fe, Ni, Zr, and their corresponding carbides, nitrides, and oxides species often overlap each other. Considering the 91 at. $\%$  Fe and the 8 at. $\%$  Ni is several times more plentiful than the 1 at. $\%$  Zr, underrepresenting the Zr content is a major potential point for counting error. This concern is amplified when we use the atom probe to describe the nano-scale oxide size and morphology, as well as volume, particle spacing and particle number density. The laser is also known to create instances of inaccuracy in C, N, and O species. A comprehensive atom probe study concerning Hf, C, N, and O, and their relationship to laser power has been conducted by Vogel et al. \cite{RN3367,RN2627}.
		
		Cameca has created a complimentary tool to the LEAP 5000 XS called the LEAP 5000 XR. The main difference between the two atom probes is that the 5000 XR utilizes as reflectron. The reflectron is an electromagnetic apparatus that changes the velocity of the moving charges in such a way that the charges are given a flight path of 382 mm compared to the 100 mm flight path in the 5000 XS. A longer flight path allows ions of different masses to have larger differences in flight times creating better mass spectrum resolutions. All atom probe data used in this dissertation were collected using a LEAP 5000 XS. One of the atom probe tips of ODS Fe\textsubscript{91}Ni\textsubscript{8}Zr\textsubscript{1}, that was ECAE processed and quasi-statically compressed, was ran on the LEAP 5000 XR at the Army Research Laboratory at Aberdeen Proving Grounds. \ref{fig:xrxs} shows the difference of mass spectra between the data collected from the 5000 XS and the 5000 XR. The data collected by the 5000 XR has lower background noise and peaks that are both more narrow and taller. Narrow peaks lead to greater accuracy and precision while ranging elements. Furthermore, the lower background exposes more peaks that would otherwise be covered. This is most evident between 30-31 Da, where the 5000 XS gives very small and broad peaks that can be easily confused as background noise. At the same range of 30-31 Da, the 5000 XR gives 3 very distinct peaks that rise out of the background. The trade-off for higher-resolution peaks is that the 5000 XR only has a 50 $\%$  detection efficiency (meaning that the detector is situated so that it will only be able to record 50 $\%$  of any ions hitting it) compared to the 80 $\%$  detection efficiency in the 5000 XS. 
		
		\begin{figure}
			\centering
			\includegraphics[width=6.27in]{fig/conclusion/xrxs.png}
			\caption[Mass-to-charge Ratio (Da) spectra of atom probe tomography of ODS\textbf{ }Fe\textsubscript{91}Ni\textsubscript{8}Zr\textsubscript{1} that has been ECAE processed at 1000 \celsius{} and quasi-statically compressed while at 200 \celsius{}.]{Mass-to-charge Ratio (Da) spectra of atom probe tomography of ODS\textbf{ }Fe\textsubscript{91}Ni\textsubscript{8}Zr\textsubscript{1} that has been ECAE processed at 1000 \celsius{} and quasi-statically compressed while at 200 \celsius{}. The blue line represents a run done with a 5000 XS and the green line represents a run done with a 5000 XR. The graphs were normalized with each other at the peak position with the most counts, 28 Da (Fe\textsuperscript{+})}.
			\label{fig:xrxs}
		\end{figure}
		
		
		
		There are a couple of studies that have studied the comparisons between a 5000 XS and a 5000 XR. A study between the 5000 XS and the 5000 XR of tungsten carbide found that the reported carbon was lower than the true carbon concentration, but the carbon content was more under reported from the 5000 XR than the 5000 XS \cite{RN3430} due to multi-hit pileup. There is a study on ODS materials between a LEAP 3000 and a LEAP 5000 that suggested that yttria particle analyses did not exhibit large differences between the two machines, where they allude to the idea that the cluster analysis techniques causes more variability in cluster reporting than the actual differences in the machines themselves \cite{RN2597}. A large interlaboratory study compared 4000 XS, 4000 XR, 5000 XS, and 5000 XR atom probes studying a geologic sample \cite{RN3444}. It has been found that the 5000 XR does give more accuracy and precision than the other instruments. There is no comparison of ODS materials between a 5000 XS and a 5000 XR. Understanding how ODS material data are collected, and how the reported size and chemistries of the ODS material changes in between the two machines could expand both atom probe technology and ODS metallurgy. During the time of writing this dissertation, it was discovered there was a porous weld on The University of Alabama’s LEAP 5000 XS main chamber where the data for this dissertation was collected. The weld was fixed resulting in consistently lower (better) vacuums than previously achieved ($<$ 5E-11 Torr instead of 9E-11 Torr). This bad weld could also be a source for the higher noise count in the 5000 XS. Additional data should be collected to confirm this.
	
	\subsection*{A Deeper Understanding of Nano-Oxide Chemistry from Processing}	
		The ODS powder that was milled with a SPEX mill, ECAE processed, and high-strain rate compression tested produced some nanoscale oxides that are slightly different from the oxides in the Zoz milled LACS sprayed powder. One of the largest differences that has not been discussed in this dissertation is the Cr shell surrounding the particle as shown in \ref{fig:proxigram}. The Cr is not present in the original powders. The most likely source of the Cr originates from the martensitic 440C stainless steel ball bearing media that mills the Fe-Ni-Zr powders. It is likely through milling that a piece of the Cr came off the ball bearings and became intermixed with the powder during milling. It is interesting that instead of mixing into solid solution, the Cr formed a shell-like structure around the nano-oxides. In the ball mill from Zoz, very little ($<$ 1 at. $\%$) Cr was detected inside of or near the particle. A Cr shell was not found. It is unclear if the Cr shell surrounding the zirconia particles have any effect on mechanical properties. A more comprehensive study between the differences in milling techniques and processing methods on oxide particles of Fe-Ni-Zr may help to further elucidate the relationships between oxide chemistry and its mechanical properties.
		
	
		
		\begin{figure}
			\centering
			\includegraphics[width=6.27in]{fig/conclusion/proxigram.png}
			\caption[Proximity histograms of Zr nano-sized clusters of ODS Fe\textsubscript{91}Ni\textsubscript{8}Zr\textsubscript{1}.]{Proximity histograms of Zr nano-sized clusters of ODS Fe\textsubscript{91}Ni\textsubscript{8}Zr\textsubscript{1}. \textbf{(a) }Powder that was\textbf{ }Zoz milled. \textbf{(b) }Milled in a shaker mill, ECAE processed at 800 \celsius{} and then compressed at high-strain rates. \textbf{(c)} Milled in a shaker mill, ECAE processed at 1000 \celsius{} and then compressed at high-strain rates.}
			\label{fig:proxigram}
		\end{figure}
		
\section*{Conclusion}
		
		This dissertation presents a collection of research pertaining to cold spray and laser assisted cold spray deposition of oxide dispersion strengthened Fe\textsubscript{91}Ni\textsubscript{8}Zr\textsubscript{1} and AISI 4340 steel. In addition to directly studying LACS of these materials, indirect studies pertaining to the mechanisms and effects of CS and LACS were also studied. Mechanisms of LACS include high-strain rate tests and other processing methods of ODS Fe\textsubscript{91}Ni\textsubscript{8}Zr\textsubscript{1} at elevated temperatures. Effects studied include residual stresses of the deposit and changes to the microstructure.
		
		A mechanically alloyed Fe\textsubscript{91}Ni\textsubscript{8}Zr\textsubscript{1} (at. $\%$) powder was fabricated through high energy ball milling. Structures were then created by depositing multiple layers through laser assisted cold spray at surface temperatures of 650 \celsius{} and 950 \celsius{}. Multi-layered deposits\ below 650 \celsius{} could not be reliably deposited. As deposit temperature increased, the grain size and nano-scale oxide clusters size increased leading to a decrease in hardness.  
		
		Laser assisted cold spray of 4340 steel produced fully consolidated deposits with a well-bonded substrate-deposit interface \textit{and} low porosity. Higher surface temperatures increased deposition efficiency and can make up for sub-optimal gas conditions. However, the combination of good gas conditions and high surface temperatures substantially increased deposition efficiency. The mechanism of deposition changed with increased surface temperature. In cold spray, the particle undergoes severe plastic deformation to achieve deposition. In laser assisted cold spray, the particle deforms less, and the substrate deforms more with increased surface temperatures. High-temperature laser assisted cold spray deposition using the current configuration leads to oxidation between steel particle interfaces, particularly in prolonged sprays. In practical applications, an inert environment may be required if laser assisted cold spray of steel is performed. Microstructure and mechanical properties are tunable with laser assisted cold spray by changing surface temperatures. Hardness of a deposit sprayed at an elevated temperature under the austenite transformation temperature is decreased due to an increase in grain size. Above the austenite transformation temperature, hardness increased despite grain growth because of martensite formation during cooling. To the author’s knowledge, this is the only study in open literature of CS or LACS deposits of 4340 steel.
		

		
	This dissertation also provided an investigation into the residual stresses produced by the LACS process on ferrous alloys. Compressive surface residual stresses on multilayered LACS deposits of ODS Fe\textsubscript{91}Ni\textsubscript{8}Zr\textsubscript{1} deposited at 650 \celsius{} and 950 \celsius{} were observed. These stresses were all compressive but were not uniform in magnitude across the deposit. In comparison, laser-only passes on the mild steel substrates used for the LACS deposition showed tensile residual stresses and martensite formation at higher surface temperatures.
	
	Quasi-static and high-strain-rate testing of the ODS materials revealed a load drop upon yielding, with the extent of the drop becoming more pronounced with increasing strain rate. Upon closer evaluation of the APT compositional data, the presence of C and N was noted at levels that make the ODS material an ultra low-carbon steel. Quasi-static tests combined with aging at elevated temperatures in between confirm that the load drops were a result of dynamic strain aging leading to the Portevin Le-Chatlier effect. 

	
	The Fe\textsubscript{91}Ni\textsubscript{8}Zr\textsubscript{1} ODS alloy was used as a case study to show how variations in isoconcentration values of atom probe tomography delineate particles which can lead to a range of compositions of those particles and varied number densities. In this work, it was found that one encompassing isosurface value is inappropriate for all desired outputs. Depending on the desired information, particle-by-particle analysis may be required to ensure the highest accuracy in the reported data. Though this may not be feasible considering the number of particles that could be present in a dataset, one should then exhibit caution in applying and reporting a common singular value for all reported quantitative outputs. When isosurface values are reported, the rationale\ for their selection is needed. 

