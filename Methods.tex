
\section*{Powder Synthesis}


	
	The ODS Fe\textsubscript{91}Ni\textsubscript{8}Zr\textsubscript{1 }was synthesized in two different ways at the Army Research Laboratory (Aberdeen Proving Grounds, MD). For the high-strain rate tests via Hopkinson-Kolsky bar experiments, the Fe\textsubscript{91}Ni\textsubscript{8}Zr\textsubscript{1} (at. $\%$) alloy was fabricated by mechanical alloying via high energy ball milling. The 99.9 $\%$  Fe, 99.8 $\%$  Ni, and 99.5 $\%$  Zr virgin powders (Alfa Aesar, Ward Hill, MA) were sieved via a -325 mesh and loaded into steel vials (SPEX model 8007) inside a glove box with an Ar atmosphere ($<$ 1 ppm oxygen). A series of 440 C stainless steel balls were used for milling with a ball-to-powder ratio of 10-to-1 by weight. After milling, the powders were stored in the glove box until roughly 100 g of powder were generated for consolidation \cite{RN740}. This process can create 10 g of ball-milled powder after 4 hours of milling, which is enough for small-scale experiments but impractical for creating larger structures through cold spray. Powders were consolidated in Ni cans at 800 \celsius{} and 1000 \celsius{} using a 4B\textsubscript{c} ECAE processing route \cite{RN140}. The final consolidated material was dense, less than 1 $\%$  porous.
	
	In order to scale powder production for cold spray use, the Army Research Laboratory then created the ODS powder by high energy mechanical alloying in a Simolayer CM08 ball mill (Zoz, Wenden, Germany) with an 8-liter capacity, which yields $ \sim $  800 grams of milled powder. The starting elemental powders were all -325 mesh sieved with the following purity levels: 99.9 $\%$  iron, 99.8 $\%$  nickel, and 99.5 $\%$  zirconium respectively. The appropriate quantities of each elemental powder were then loaded into a steel vial with 440C stainless steel balls, at a 10:1 ball to powder ratio, for high energy ball milling in an argon atmosphere at 400 rotations per minute for 30 hours yielding the desired composition and microstructure. The powders during this milling process were cooled to -25 \celsius{} to prevent powder (cold) welding during milling. 

	The AISI 4340 steel was created through commercial gas atomization (Carpenter Technologies, Philadelphia, PA). 


\section*{Mechanical Testing}

	
	
	Powders were consolidated using ECAE processing for mechanical testing. Tensile dog bone samples were cut from ECAE processed material at 1000 \celsius{}. Compression testing samples were created from material that was processed at 800 \celsius{} and 1000 \celsius{}. Mechanical testing was graciously performed by Prof. Kiran Solanki and Chaitanya Kale at Arizona State University. Compression testing was performed at quasi-static rates (10\textsuperscript{$-$ 3} s\textsuperscript{$-$ 1}) and high strain rates (4$ \times$ 10\textsuperscript{3} s\textsuperscript{$-$ 1}) at room temperature ($ \sim $ 298 K) and 473 K, with the temperature provided by a furnace that encapsulated the specimen and the major testing parts of each apparatus. In both cases, the system was held at the testing temperature for about 30 min to attain thermal equilibrium, with a thermocouple attached on the specimen to monitor the temperature. Quasi-static uniaxial compression and compression loading-unloading tests were performed on a load frame (Instron, Norwood, MA) with a 50 kN load capacity while high rate experiments were performed on a 12.7 mm Inconel Kolsky bar apparatus \cite{RN1012,RN1053}. 



\section*{Laser Assisted Cold Spray}


	
	Multi-layered structures of ODS Fe\textsubscript{91}Ni\textsubscript{8}Zr\textsubscript{1 }and AISI 4340 steel were LACS deposited. The cold spray system at The University of Alabama is a system combining both a Gen III high-pressure cold spray system (VRC Metal Systems, Rapid City, SD) with a Viper robotically controlled powder-gas applicator and nozzle. A powder and heated gas line are connected to the applicator. The gas heater (maximum 21 kW) is in the booth (\ref{fig:Methods1}b). The powder line is connected to a powder feeder. The powder feeder is a steel cylinder with a drum inside (\ref{fig:Methods1}d). The drum has evenly spaced holes near the edge of the drum. Attached to the lid is a gas line connected to a spring system such that the gas line is pressed flush with the drum while it is closed (\ref{fig:Methods1}e). The gas line/plunger system is at the same location of the holes on the drum such that as the drum rotates, powder is caught up in the gas and goes through the hole in the drum that is aligned with the output line of the powder feeder (\ref{fig:Methods1}f). Attached to all of this (not shown) are pressurized cylinders of gas connected to a series of mass flow controllers regulating the pressure and flow rates of gas to the system. This system allows for control over powder feed rates, gas pressure and temperature, local substrate surface temperatures, and spray geometries.
	
	An LDM-4000-100 variable power (maximum 4 kW) diode laser (Laserline, Mülheim-Kärlich, Germany) was combined with this system. \ref{fig:Methods1}a shows inside of the LACS deposition enclosure with the laser irradiating a substrate. The laser is generated outside of the spray downdraft box with laser light routed to an optics head that is attached to the powder-gas applicator with the nozzle below it. The optics head is fixed such that the laser is angled 60.9 $ ^{\circ} $  with respect to the substrate with a spot diameter of 8 mm \cite{RN720}. 


	
	
	\begin{figure}
		\centering
		\includegraphics[width=6.27in]{fig/Methods/Picture1.png}
		\caption[Photographs of the laser assisted cold spray process.]{Photographs of the laser assisted cold spray process. \textbf{(a) }LACS setup with the laser, power-gas applicator, nozzle, and the substrate with deposition. \textbf{(b) }Gas heater and separate powder and gas lines. \textbf{(c) }Diode laser with fiber optics cable leading to the laser head. \textbf{(d) }Powder feeder drum with holes. \textbf{(e) }Powder feeder lid with gas connection input and powder selector \textbf{(f) }Bottom of the powder feeder with an output port.}
		\label{fig:Methods1}
	\end{figure}
	


\section*{Microstructural Characterization}

	\subsection*{X-ray Diffraction}	
		X-ray diffraction of the powders were collected using a G-8 GADDS system (Bruker, Billerica, MA) with a Co source. 
		
		Surface residual stress was measured using an iXRD (Proto Manufacturing, MI, USA) x-ray residual stress diffractometer. The residual stresses parallel to the traversal path of the deposit were measured. These measurements were collected using a Cr k\textsubscript{$ \alpha $ } x-ray source (wavelength $ \lambda $  = 2.291 nm, 1 mm diameter circular beam) and employed the d versus sin\textsuperscript{2}($\psi$) approach in conjunction with elliptical data fitting to determine the in-plane ($\sigma$) and out-of-plane ($\tau$) residual stress components at each point. The hkl plane measured was (211) which has a Bragg angle of 156.31 $ ^{\circ} $  and d-spacing, d\textsubscript{0} = 1.1704 Å. The system alignment was checked using a stress-free iron powder (residual stress of 0 $ \pm $  14 MPa) and high-stress steel standard (residual stress of -474 $ \pm $  35 MPa) for each set of measurements. Two detectors, fixed on opposite sides of the x-ray source (to simultaneously measure negative and positive  tilts), were incrementally rotated in the  orientation to measure dspacing as a function of sin\textsuperscript{2}($\psi
		$) as illustrated in \ref{fig:Methods2}. 
		
		
		\begin{figure}
			\centering
			\includegraphics[width=6.27in]{fig/Methods/Picture2.png}
			\caption[Schematic of the orientations of XRD in the case of determining residual stresses.]{Schematic of the orientations of XRD in the case of determining residual stresses. The directions $ \theta$, $ \varphi$, and $ \psi$ are orthogonal to each other.}
			\label{fig:Methods2}
		\end{figure}
		
		
		
		Previously derived equations may be used to determine the strain in the material \cite{RN1385}. When Hooke’s law is applied:
		
		
		\begin{equation}		
			\sigma _{ij}=C_{ijkl} \epsilon _{kl},
		\end{equation}
		
		
		
		determined strain may be written as a function of stress and materials properties:
		
		
		\begin{equation}
			\epsilon _{ \phi  \psi }=\frac{ \left( d_{hkl} \right) _{ \phi  \psi }-d_{0}}{d_{0}}=\frac{1+ \nu }{E} \left(  \sigma _{ \phi }sin^{2} \psi  \right) -\frac{ \nu }{E} \left(  \sigma _{1}+ \sigma _{2} \right),
		\end{equation}
		
		where \textit{$ \sigma $\textsubscript{$ \varphi $  }}is the surface stress at angle \textit{$ \varphi $ }, \textit{E} is the modulus of elasticity, \textit{$ \nu $ } is Poisson’s ratio, and \textit{$ \sigma $ }\textsubscript{1} and \textit{$ \sigma $ }\textsubscript{2} are the principal stresses. Equation 2 then may be written as:
		
		
		
		\begin{equation}
			d_{ \psi }=\frac{1+ \nu }{E}d_{0}  \sigma _{ \phi }sin^{2} \psi  -\frac{ \nu }{E} (  \sigma _{1}+ \sigma _{2} ) d_{0}+d_{0}.	
		\end{equation}
		
		
		
		The variation of the plane spacing with  \( sin^{2} \psi  \)  is linear, with the slope represented as:
		
		
		
		\begin{equation}
		m=\frac{1+ \nu }{E}d_{0} \sigma _{ \phi }.
		\end{equation}
		
		
		
		From the known materials properties and the slope of d-spacing and sin\textsuperscript{2}($\psi$), residual stress can then be calculated. This is known as the sin\textsuperscript{2}($\psi$) method of determining residual stress through XRD \cite{RN1385}. The method of calculating residual stress, as well as machine alignment through stress-free and pre-stressed material, has been standardized (see ASTM: E915-19 and ASTM: E2860-12). For determining residual stress, the elastic constant used was: 
		
		
		\begin{equation}
			\frac{E}{1+ \nu }=169 \; GPa.
		\end{equation}
		
		In the case of compressive in-plane stresses, the measured d-spacing will decrease with increasing sin\textsuperscript{2}($\psi$), generally in a linear trend, with the opposite effect occurring for tensile stress conditions. In a sample portion with no stresses in the surface regions, the measured dspacing remains constant at all  angles. i.e. zero gradient.
		
	\subsection*{Microscopy}	
		
		Starting powder, mechanically tested samples, and LACS deposits were cut and mounted. Samples were grinded with successively finer grits (320-1200). The samples were then polished with diamond suspensions at 9 $ \mu$m, 3 $ \mu$m, and 1 $ \mu$m. Samples were polished with 0.05 $ \mu$m colloidal silicon or alumina and then vibratory polished with 0.02 $ \mu$m colloidal silicon.
		
		It was unusually difficult to achieve a polish good enough quality to be used by available EBSD camera technologies. CS deposits experience high amounts of strain that make it difficult to achieve the detectable Kikuchi bands. Careful and normal polishing procedures for 4340 (including an alumina final vibratory polish) was sufficient for EBSD collection. In the case of the ODS material, polishing for EBSD was more difficult. The ODS material starts as nanocrystalline. Although the grains do increase in size after cold spray and laser assisted cold spray, they still are less then 5 $ \mu$m in diameter. The small grain diameters in combination with the severe strains make for difficult polishing. Because of this, after a 0.05 $ \mu$m alumina polish, a 0.02 $ \mu$m colloidal silica polish for 4 hours was applied. This causes etching at the interfaces and makes for poor optical microscopy but allows for a good enough polish for EBSD. Microscope images were not polished the same depending on which microscopy technique use. 
		
		Optical micrographs were obtained using an inverted metallurgical microscope (Amscope, Los Angeles, CA). Scanning electron micrographs were obtained used an SEM-FIB Lyra (Tescan, Brno, Czech Republic), an FE-7000 (JEOL, Tokyo, Japan), and an Apreo, (Thermo-Fischer, Waltham, MA). Focused ion channeling contrast images were obtained using an SEM-FIB Quanta (Thermo-Fischer) at 30 kV and 30 pA. EBSD scans were performed on three different machines: a FE-7000, an Apreo, and Tescan. The FE-7000 used an AZtec channel 5 system (Oxford Instruments, Abingdon, United Kingdom). The Apreo and Tescan used Octane Elite cameras (EDAX, Draper, UT). EBSD data collected through EDAX machines were analyzed through the OIM analysis software EBSD and EDS scans were taken at 20 kV unless indicated otherwise with step-sizes less than 100 nm. Data were cleaned down to 4 nearest neighbors.
		
		
		The nanoscale structure and composition of the ODS Fe\textsubscript{91}Ni\textsubscript{8}Zr\textsubscript{1} material was further analyzed using local electrode atom probe tomography (Cameca, Madison, WI) \cite{RN704}. Atom probe samples were created through the lift-out method in an SEM-FIB Quanta \cite{RN347}. A 1 $\mu$m protecting Pt cap layer was deposited through a gas infiltration system. Trenches were milled at an angle of 30 \celsius{}  to the surface where the suspended sample was welded to a small Omniprobe needle (Oxford Instruments), taken out, and welded to flat top Si posts. The Pt deposition current was at 0.1 nA. The milling has been done at both 1 nA and 3 nA. As far as Ga\textsuperscript{+} ion contamination is concerned for the ODS materials, there is no obvious difference if 3 nA is used for the entire milling, 3 nA is used with a 1 nA closer cut is used or 1 nA is used for the entire milling process. Images, however, must be taken one at a time (not a continuous scan) and most importantly, the milling cannot drift over the Pt bar. The Pt bar appears to protect against stray ions, but constant milling over the Pt bar, even for a few seconds will cause $>$ 10 $\%$  Ga\textsuperscript{+} contamination in the atom probe samples. Should milling drift over the bar milling at this location should be stopped and the process restarted at a different location. Because the strength of the ODS material, expect milling times to be about 2x the amount of time required than for a softer material such as Al alloys. 
		
		
		
		Lifted-out and mounted samples were then sharpened with a SEM-FIB Tescan. Milling was done at 30 kV with three annular polishing steps at decreasing currents: 200-250 pA, 100-120 pA, and 50-60 pA until a tip radius of 100 nm was measured 200 nm from the top as measured with \textit{in situ} back scattered electron (BSE) imaging always at 130 kx magnification. BSE imaging is more element dependent than surface dependent. It is important to try and have the Pt cap present on top of the tip until the final milling step. Brightness and contrast may need to be adjusted when the Pt cap disappears as the high atomic Z number causes a much higher brightness than the rest of the tip. Even though secondary electron imaging during milling may be useful because it has better resolution and can see a truer image of the material, it is the author’s personal opinion that the milling process interferes with the electrons too greatly to gain the benefits of better resolution. 
		
		
		
		Atom probe reconstruction is based off of shank angle and initial tip of the radius. Because of the rapid milling of the tip at the last step, adjusting a tip radius and shank angle requires too much time, and the region of interest may be consumed before proper measurements are made. The 100/200 ratio is much easier to manage and produce reproducible tips. The SEM resolution is too poor to detect the actual tip shape. In addition, at these thicknesses, the tip starts to become electron transparent and more difficult to detect. If exact tip shape is desired, it should be imaged by a transmission electron microscope. Accurate and precise measurements of the atom probe tip using SEM technology are unreasonable but maintaining a visible tip ratio has proven to be a reproducible way to manufacture tips and compare evaporation characteristics. 
		
		
		
		A final cleaning step at 5 kV and 60-80 pA was performed to reduce surface Ga ion implantation. With this material, the final polishing step does not appear to change the tip shape too much. The tip was cleaned at this voltage and current until about 50 nm of the material is removed. While the high strength of the material makes for difficult milling, sharpening is considerably easier because the glancing ions on the angle smooths the cone surface easily and Ga\textsuperscript{+} is less of a concern. After the final polish, the sample was removed from the microscope and immediately loaded into the atom probe loading chamber. 
		
		
		
		Atom probe tips were cooled to 40 K at a vacuum of $<$ 10\textsuperscript{-8} Pa, brought to a continually increasing direct current voltage, and field evaporated with a pulsing laser (rates between 300-600 kHz) at 100 pJ. Voltage was kept such that each laser pulse has a 1.0 $\%$  chance of producing a detectable field evaporation event per pulse (listed as detection efficiency). Data were reconstructed and analyzed by IVAS 3.8 (Cameca). With the oxides changing the required voltage for evaporation, voltage mode reconstruction seemed unreliable. Shank-angle mode was used with parameters changed through a guess-and-check method until the most spherically shaped oxides were produced. Mass spectrum ranging proved to be especially difficult and explained in chapter 3. The greatest cause for frustration was that Fe, Ni, and Zr peaks (especially the C, N, and O species of the metal) overlap constantly. While there is technology that does peak deconvolution, the background combined with the multitude of overlapping peaks overwhelmed current peak deconvolution methods and did very little to improve the error caused by ion selecting. Cluster analysis was performed using both isoconcentrations determined by proximity histograms and nearest neighbor approaches. Both cluster counting techniques have strengths and weaknesses which will be illustrated in Chapter 5.
