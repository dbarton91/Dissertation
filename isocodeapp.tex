\chapter{CODE LISTINGS FOR ATOM PROBE RANDOMIZING SIMULATIONS}

\section*{Chemical Randomizing Code}

	The first method is keeping the same positions and randomizing the mass-to-charge positions. The file format is written as .pos converted to .csv files using IVAS. These files have four columns: x position, y position, z position, and mass-to-charge ratio. Changing a file of 75 million ions in this manner took about 30 minutes with a 2.2 GHz processor. This is the code written in Python 3.6 and ran on a UNIX system, MacOS.


{\setstretch{1.0}
\lstinputlisting[language=Python,caption=Chemical Randomizing Code]{Isosurfaces/Code/chemicalrandomizing.py}	
}


\section*{Complete Spatial Randomness Code}

The code for complete spatial randomness is based off of the equation found in \cite{RN1114},
\begin{equation}
\label{isoappeq1}
 P_{K}(r,p)dr = \dfrac{3}{(K-1)!}\left(\dfrac{4\pi}{3}\rho\right)^{K}r^{3K-1}exp\left(-\dfrac{4\pi}{3}\rho r^{3}\right)dr. 
\end{equation}

Large exponents and factorials require too much computational power. A natural log was applied to the equation eliminating the exponent and allows for Stirling’s approximation to be used.

{\setstretch{1.0}
	\lstinputlisting[language=Python,caption=Complete Spatial Randomization Code]{Isosurfaces/Code/CSR.py}	
}