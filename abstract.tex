Oxide dispersion strengthened (ODS) Fe\textsubscript{91}Ni\textsubscript{8}Zr\textsubscript{1} (at. $\%$) and AISI 4340 steel were successfully deposited via laser assisted cold spray. The laser assisted cold spray technique includes accelerating powder particles such that they strike a substrate at supersonic speeds and metallurgically bond. A high-powered laser irradiated the surface of the deposition area making the substrate surface thermally softer promoting deposition.

\textit{In situ} laser heating of the substrate increased the deposition efficiency of the high strength 4340 steel from 48 $\%$  to 72 $\%$. The increased surface temperature from 400 \celsius{} to 950 \celsius{} also increased the median ferrite grain size. As the ferrite grain size increased, the hardness decreased; however, at higher surface temperatures, the steel transitioned to martensite and compensated the lost hardness due to grain size with the hardness returning to the same values as the cold spray deposits. 

The lowest viable surface temperature achieved for multi-layered LACS deposition of ODS materials is 650 \celsius{}. Increased surface temperatures led to an increase in deposition efficiency up to 32 $\%$ at 950 \celsius{} and resulted in a lower hardness material. Grain sizes and particle sizes increased from the elevated temperatures as well. However, the grains did not grow the same throughout the thickness of the material. Grains near the surface of the deposit are several times larger than grains near the deposit-substrate interface.

In addition to LACS processing and microstructure, this work reports compressive surface residual stresses of LACS deposits, dynamic strain aging of ODS materials, and a commentary for improving cluster analysis of nano-scale oxides measured through atom probe tomography.
   
%
%
%Mechanically alloyed Fe\textsubscript{91}Ni\textsubscript{8}Zr\textsubscript{1} (at. $\%$) powders were fabricated through high energy ball milling of elemental powder and subsequently consolidated via equal channel angular extrusion (ECAE) at 800 \celsius{} and 1000 \celsius{}. The resulting microstructure was fine grain with a nano-dispersion of Zr-oxide within the matrix, which was spherical for the 800 \celsius{}. ECAE and plate-like (and volumetrically larger) for the 1000 \celsius{} ECAE conditions. Atom probe tomography confirmed trace levels of C, N, and Cr impurities within the alloy making it similar to a low-carbon steel. By performing mechanical testing at a quasi-static strain rate (10\textsuperscript{-3} s\textsuperscript{-1}) and at high strain rates (10\textsuperscript{3 }s\textsuperscript{-1}) at room temperature and 473 K, a load drop was noted after yielding. In general, this load drop became more pronounced with increasing strain rate and temperature and has been shown to be a result of dynamic strain aging in the ODS alloy. 
%
%
%Residual stresses of laser assisted cold spray deposition of an iron-based oxide dispersion strengthened alloy (Fe\textsubscript{91}Ni\textsubscript{8}Zr\textsubscript{1 }at.\ $\%$)\ were measured  as well as its underlying AISI 1018 mild steel substrate. X-ray diffraction-based measurements were used to map the residual stress values. The compressive residual stresses ranged from -170 to -440 MPa for a 650 \celsius{} and 950 \celsius{} surface temperature deposit at a raster deposition rate of 25 mm/s.
%
%
%Isoconcentration surfaces are commonly used to delineate phases in atom probe datasets. These surfaces then provide the spatial and compositional reference for proximity histograms, the number density of particles, and the volume fraction of particles within a multiphase system. The zirconia particles were identified by varying the Zr-isoconcentration values as well as by the maximum separation data mining method. The associated outputs are elaborated upon in reference to the variation in this Zr isosurface value.
%
%Laser assisted cold spray (LACS) of AISI 4340 steel, with sample deposited thicknesses between 1.7 mm and 3.5 mm, was accomplished by combining an infrared (940 nm wavelength) 4 kW laser with a VRC Metal Systems generation III cold spray instrument. \textit{In situ }laser heating of the substrate increased the deposition efficiency of the high strength 4340 steel from 48 $\%$  to 72 $\%$. The increased surface temperature from 400 \celsius{} to 950 \celsius{} also increased the median ferrite grain size from 0.51 $ \mu $m to 2.2 $ \mu $m. As the ferrite grain size increased, the hardness decreased from 561 HV to 466 HV; however, at higher surface temperatures (738 \celsius{}\ and 950 \celsius{} the creation of martensite occurred and compensated the lost hardness (i.e., due to grain size) with the hardness returning to values of 532 HV and 592 HV respectively. In addition, LACS with increased laser power changed the impact behavior of particles to substrate penetration instead of typical plastic deformation observed in regular cold spray process.  Increased laser power also resulted in micro-scale oxidation during deposition.
