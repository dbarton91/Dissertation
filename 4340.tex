\chapter{LASER ASSISTED COLD SPRAY OF AISI 4340 STEEL}


Note: This paper is currently under review for publication in a peer-reviewed journal.

\section*{Abstract}

Laser assisted cold spray (LACS) of AISI 4340 steel, with sample deposited thicknesses between 1.7 mm and 3.5 mm, was accomplished by combining an infrared (940 nm wavelength) 4 kW laser with a VRC Metal Systems generation III cold spray instrument. \textit{In situ} laser heating of the substrate increased the deposition efficiency of the high strength 4340 steel from 48 $\%$ to 72 $\%$. The increased surface temperature from 400 \celsius{} to 950 \celsius{} also increased the median ferrite grain size from 0.51 $\mu$m to 2.2 $\mu$m. As the ferrite grain size increased, the hardness decreased from 561 HV to 466 HV; however, at higher surface temperatures (738 \celsius{} and 950 \celsius{}) the creation of martensite occurred and compensated the lost hardness with the hardness returning to values of 532 HV and 592 HV respectively. In addition, LACS with increased laser power changed the impact behavior of particles from particle flattening to substrate penetration. Increased laser power also resulted in micro-scale oxidation during deposition.



\section*{Introduction}


AISI 4340 steel \cite{RN2268}, a heat treatable, low alloy steel, is widely used in structural applications due to its high ductility and fracture toughness in addition to having a high yield strength \cite{RN1418}. AISI 4340 steel has alloy contents of Cr, Mo, Ni, Mn, and Si for a total alloy content of less than 4$\%$. The tailored composition of 4340 steel, particularly Ni, significantly increases bainitic and pearlitic transformation times while cooling. This delay allows for greater control over the martensite/ferrite phase fraction therefore influencing the mechanical strength \cite{RN2269}. For example, high phase fractions of martensite can lend itself to high strength and uniform hardenability. This ability to have a tunable microstructure has enabled 4340 steel to be used in a wide variety of applications that includes crankshafts in maritime engines and aircraft components \cite{RN3385,RN1427}.



Because of the extensive use of 4340 steel in large structural components, a method to conduct on-site repairs is required. Employing commonly used fusion welding facilitates melting of material with an accompanying rapid solidification that can create microstructure differences from the original 4340 steel microstructure resulting in unwanted strength differences, and in some cases, hydrogen cracking by an excess formation of martensite \cite{RN771}. Although this issue can be mitigated by pre-heating and post-heat treating of the alloy, a solid-state repair technique that can avoid excessive heat input and the possibility of hydrogen cracking is desirable.



One method for solid-state repair is cold gas dynamic spray deposition, commonly shortened to $``$cold spray$"$  (CS). The cold spray technique feeds fine powder with heated and pressurized gas through a de Laval nozzle. Particles accelerate to supersonic speeds and then impact the substrate resulting in severe plastic deformation. The impact deformation causes the material to create a solid-state bond with the substrate and any subsequently deposited layers. By continually depositing the material in a layer-by-layer process, a larger structure is formed. 



Most CS research has been in materials with relatively low melting points and low mechanical strengths, including Al, Ni, and Cu \cite{RN706}. Materials with high strengths and high melting points, as well as body-centered cubic (BCC) materials, are regarded to be more difficult to CS because of the requirement for high deformability at high strain rates \cite{RN3375}. Despite these challenges, some success using high pressure cold spray has been reported for niobium and tantalum \cite{RN3445,RN1938}.  In addition there are recent reports for successful deposition of other high strength ferritic/martensitic steels such as 300 Maraging Steel \cite{RN3404}. Successful cold spray deposition of 4340 has not yet been reported in the open literature.



One way to achieve improved deposition of such ‘difficult materials’ is to increase the surface temperature during the deposition, where materials become reasonably ‘softer’ and easier to deform. Bray et al. introduced a technique to achieve this increase in surface temperature during Ti deposition by using a laser \cite{RN173}. The laser locally heats the substrate directly under the nozzle. A change in the surface temperature of only a 100 \celsius{} created a notable difference in the deposit quality. This use of a laser is referred to as either laser assisted cold spray (LACS) or supersonic laser deposition. Besides Ti, additional difficult-to-deposit materials such as Co-based Stellite and an oxide dispersion strengthened alloy have also been reported to be deposited using LACS \cite{RN135,RN1390,RN383}. Furthermore, recent work on low pressure cold spray of Cu and Ni with additions of alumina powders, have also shown the benefit of \textit{in situ} laser heating.  These cermets were successfully cold sprayed with assistance from laser heating at a gas pressure and temperature of 0.6 MPa and 445-650 \celsius{}, which results in low particle velocity condition \cite{RN3366}.


This paper expands on these prior studies by now demonstrating CS and LACS deposition of 4340 steel powder with and without \textit{in situ} laser heating of the substrate during deposition. The sensitivity of ferritic, austenitic, and martensitic phases in 4340 steel with temperature and cooling rates provides an ideal material to further the investigation of LACS. While there have been reports of CS of nickel chrome-chrome carbide and Fe-based metallic glass particles deposited on 4340 steel substrates \cite{RN511,RN345}, there is still little information concerning CS and LACS of 4340 steel powder deposits. By adjusting the substrate temperature during deposition, via using the \textit{in situ} laser heating, the influence of high-pressure cold spray (HPCS) and laser heating parameters on the deposition characteristics of 4340 steel will be determined.



\section*{Experimental Procedure}

\subsection*{Sample Creation}
\label{Sample Creation}


AISI 4340 steel powder (Carpenter Technologies, Philadelphia, PA) was produced using gas atomization. Cold spray was performed using a Gen III HPCS system (VRC Metal Systems, Rapid City, SD). The gas was set at different temperatures and pressures as measured at the gas-powder applicator directly above the nozzle. The cold spray de Laval nozzle used was a standard WC nozzle (VRC Metal Systems $\#$ 60), 200 mm in length with a 2 mm throat diameter and an exit diameter of 6.3 mm. The nozzle and spray plume direction were kept normal to the substrate. The standoff distance of the nozzle from the substrate was 25 mm. The spray nozzle was controlled by a Viper robotics system (VRC Metal Systems) coded in CNC programming. The transverse direction was 27 mm. Longitudinal spacing between the transverse lines were 1 mm apart. There were 36 passes resulting a final rectangle of 27 x 36 mm. After a raster was completed, the nozzle would turn around and repeat the same raster in reverse laying down an additional layer on top of the cold-spray deposit. Six layers were deposited in this way. The nozzle moved up in the z direction with 0.1 mm between each layer. Individual substrates were used for each deposit. Substrates were AISI 1018 steel plate of 75 mm (width) x 300 mm (length) x 12.6 mm (thickness). The steel plates were sanded with a P-80 grit sandpaper, then washed with soapy water and rinsed with isopropyl alcohol. The rough grit was used to increase the absorption of the laser. The same substrate preparation method was used for CS without laser heating in order to keep the same surface conditions. 


An LDM-4000-100 variable power (maximum 4 kW) diode laser (Laserline, Germany) was used during the deposition. The angle of the laser to the substrate was 60.9 $ ^{\circ} $ . The laser was fixed to the powder/heated gas applicator such that as the nozzle moved, the laser would move with it keeping a constant laser spot underneath the nozzle \cite{RN511}. In order to control the temperature of the surface, a Mergenthaler unicolor pyrometer with a closed loop feedback system to the laser was employed in conjunction with the irradiating laser. The feedback control system changes laser power throughout the run in order to keep the measured surface temperature constant. A photograph of the LACS setup is shown in \ref{fig:43401}.


\begin{figure}
	\centering
	\includegraphics[width=6.27in]{fig/4340/Picture1.png}
	\caption{Photograph of the laser assisted cold spray with laser irradiation of the substrate.}
	\label{fig:43401}
\end{figure}


LACS deposits were created at a variety of different surface temperatures in order to represent the three common crystal structures and combination of crystal structures. The BCC to face-centered cubic (FCC) crystal structure transition temperature for AISI 4340 steel is reported to begin between A\textsubscript{c1} = 710-727 \celsius{} with a complete transition to FCC occurring between A\textsubscript{c3 }= 770-927 \celsius{} \cite{RN3376,RN3377,RN3379,RN3380,RN3382}. BCC+FCC crystal structures should exist simultaneously between the A\textsubscript{c1} and A\textsubscript{c3} ranges of temperature. The surface temperatures at which LACS deposits were performed are as follows: 


\begin{itemize}
	\item Regular CS without the laser heating. \par
	
	\item Laser heating of the substrate to 500 \celsius{}, which will keep the steel completely in the ferritic (BCC) zone.\par
	
	\item Laser heating up to 738 \celsius{}, allowing for some combination of BCC and FCC phases to be present.\par
	
	\item And laser heating to 950 \celsius{}, which is sufficiently high enough temperature to nominally transition completely to FCC.
\end{itemize}\par


The deposition efficiency was determined as the mass difference of the substrate divided by mass difference of the used powder.


\subsection*{Characterization}


Samples for microscopy characterization were prepared by cross-sectioning the center of the deposit normal to a raster direction, mounting through resin embedding, grinding with successively finer grit paper down to P-1200, and then polishing with 9 $ \mu $m, 3 $ \mu $m, and 1 $ \mu $m diamond suspension. The samples were vibratory polished with a 0.05 $ \mu $m alumina solution for 2 hours.

Deposit and interface quality were determined through optical microscopy (Inverted Metallurgical Microscope, AmScope, CA). Back-scattered electron images were obtained with an FE-7000 scanning electron microscope (JEOL, MA) and Lyra 3 scanning electron microscope (Tescan, PA), with energy dispersive spectroscopy (EDS) scans using an Octane Elite Super (EDAX, UT) and electron back scattered diffraction (EBSD) detectors at 20 kV with step sizes of 100 nm or smaller. The corresponding data was analyzed on the Oxford AZtec platform with the Channel 5 analysis software (Oxford Instruments, MA). Both the BCC and FCC iron crystallographic structures were used to index the Kikuchi patterns. A nearest neighbor cleanup step (down to 4 nearest neighbors) through the software Tango (Oxford Instruments, MA) was implemented on the EBSD data. Grain size calculations were make after the cleanup step.

\section*{Results}


\subsection*{Starting Powder}


The starting powder was produced through gas atomization and shows traits that are commonly seen in this form of powder processing. The powder is mostly spherical with some satellites, i.e. smaller powder particles that have attached themselves to larger powder particles without fully coalescing. Powders were mounted and polished, then measured for size. The median particle size has a diameter of 33 $ \mu $m (\ref{fig:43402}b) with the largest measured particle at 80 $ \mu $m diameter. The distribution was log normal. Due to the rapid solidification and fast cooling rates of the powders, large volume fractions of martensite are present in the powder as seen through the lenticular shaped phase boundaries detectable through back-scattered electron imaging, a sample of which is shown in \ref{fig:43402}c. White arrows in the image show structures indicating martensite.

\begin{figure}
	\centering
	\includegraphics[height=7in]{fig/4340/Picture2.png}
	\caption[Gas atomized 4340 steel powder.]{\textbf{(a)} Electron micrograph of gas atomized 4340 steel powder, \textbf{(b)} Effective cumulative volume fraction of powders as a function of particle diameter, \textbf{(c)} Back-scattered electron micrograph of a mounted and polished powder particle. White arrows are pointing to martensitic structures.}
	\label{fig:43402}
\end{figure}


\subsection*{Cold Spray and Laser Assisted Cold Spray deposits}

Macroscale images of the CS and LACS deposits are shown in \ref{fig:43403}. The cone-shaped structures are powder deposits from extra gas blowing through the lines to cool down the heater after the main run. These images show that the radius of coloration caused by oxidation of the steel increased with higher fixed surface temperature. Deposit thicknesses increased from 1.7 mm with just CS (no laser) up to 3.5 mm with laser heating at a 950 \celsius{} surface temperature condition. The three lowest temperature conditions have a rough surface noted by the small dimples present. At the surface temperature of 950 \celsius{}, the dimples are no longer present and individual raster lines are observable creating a ribbed surface texture of the material. In all four deposits, no cracks or signs of delamination are present.



\begin{figure}
	\centering
	\includegraphics[width=6.27in]{fig/4340/Picture3.png}
	\caption[Photographs of LACS deposits of 4340 steel.]{Photographs of \textbf{(a)} CS (no laser), and LACS deposits at surface temperatures of \textbf{(b)} 500 \celsius{}, \textbf{(c)} 738 \celsius{}, and \textbf{(d)} 950 \celsius{}. A copper penny is placed next to the deposit for reference of size.}
	\label{fig:43403}
\end{figure}



The mass-based deposition efficiency clearly increased with increasing surface temperature (\ref{fig:43404}). The deposition efficiency showed a linear increase with surface temperature until the 738 \celsius{} condition. The blue square denotes LACS deposition at a gas temperature and pressure of 475 \celsius{} and 3.10 MPa. The red square denotes LACS deposition at a gas temperature and pressure of 550 \celsius{} and 4.14 MPa. The open square in the figure designates a cold spray deposit with no laser. Pyrometer measurements of the surface during cold spray deposition \underline{without the laser heating} records a temperature of approximately 400 \celsius{}.

\begin{figure}
	\centering
	\includegraphics[width=6.27in]{fig/4340/Picture4.png}
	\caption{Deposition efficiency according to different gas conditions and measured surface temperatures.}
	\label{fig:43404}
\end{figure}


\subsection*{Microstructural Evolution with Temperature}


The CS and LACS deposits were fully dense with porosities of less than 1 $\%$  in all the samples (\ref{fig:43405}). The porosity was found to be slightly higher in the CS deposit than in the LACS deposits with complete bonding between the deposit and the substrate \textit{can be} noted for all samples. As the surface temperature increased, individual layers in the deposits became more easily demarcated and the deformation of the layer underneath each spray more discernable. 


\begin{figure}
	\centering
	\includegraphics[height=7in]{fig/4340/Picture5.png}
	\caption[Optical micrographs of CS and LACS deposits of 4340 steel.]{Optical micrographs of \textbf{(a)} CS (no laser) and LACS deposits without the interface at surface temperatures of \textbf{(b)} 500 \celsius{}, \textbf{(c)} 738 \celsius{}, and \textbf{(d)} 950 \celsius{}. The top row shows the deposit where the surface is visible. The bottom row shows the substrate-deposit interface as indicated by the black arrows. There may be overlap between the top and bottom micrographs.}
	\label{fig:43405}
\end{figure}


The morphology of impact particles at the deposit/substrate interface reveals the effect of the \textit{in situ} laser heating (\ref{fig:43406}). In these backscatter electron images, the lighter part of the image is the 4340 steel, while the darker portion of the images are the 1018 steel substrate. The CS sample has highly deformed, flattened particles commonly found in other micrographs of CS deposits. Recall that the starting particles were nearly spherical in shape (\ref{fig:43402}a). The CS deposit micrographs also show that the 4340 steel powder experienced substantial deformation while the substrate was not readily deformed. This is evident in \ref{fig:43406}a where the bottoms of the particles are flat, and the substrate is not deformed. At the 500 \celsius{} surface temperature, there was little noticeable change of particle shapes from the CS deposit. Arrows point to the bottoms of the particles in \ref{fig:43406}a and \ref{fig:43406}b for reference. At the 738 \celsius{} surface temperature condition, the powder particles deformed, but the substrate deformed as well, so that the bottoms of the impacted particles maintained some of their roundness. As the surface temperature increased further, the powder particles implanted themselves deeper into the substrate. At the 950 \celsius{} surface temperature, nearly undeformed particles can be observed fully embedded into the substrate and is pointed out by an arrow (\ref{fig:43406}d). This change in deformation is most likely because of the combination of both the high pressures at the impact zone \cite{RN1395} combined with the considerable decrease in strength of the mild steel substrate at the elevated temperature. The dark features in \ref{fig:43406}c and \ref{fig:43406}d are oxides formed during the LACS deposition. The roundness of the bottom of the particle increased with increasing surface temperature due to substrate softening. The effect of substrate and layer softening shown in \ref{fig:43406} is mapped out across temperatures and deposit position. The top, middle, and bottom portions of the deposits at each condition are shown in the BSE micrographs of \ref{fig:43407}. One can glean from these images an increased presence of a dark contrast linear feature in the deposit, which appears to outline some of the deposited grains (particles). This dark contrast is produced by an oxide, which has been confirmed by energy dispersive spectroscopy in \ref{Deposition Mechanisms of LACS}. The amount of oxygen content increased both as a function of surface temperature and its spatial location being closer to the deposit-substrate interface. Some oxygen regions are also indicated in \ref{fig:43407} by the arrows.


\begin{figure}
	\centering
	\includegraphics[width=6.27in]{fig/4340/Picture6.png}
	\caption{Electron micrographs of the deposit-substrate interface of \textbf{(a)} CS (no laser) and LACS with surface temperatures of \textbf{(b)} 500 \celsius{}, \textbf{(c)} 738 \celsius{}, and \textbf{(d)} 950 \celsius{}.}
	\label{fig:43406}
\end{figure}


\begin{figure}
	\centering
	\includegraphics[width=6.27in]{fig/4340/Picture7.png}
	\caption[Series of electron micrographs near the top, middle, and bottom (but above the deposit-substrate interface) of the \textbf{(a)} CS (no laser) deposit, and LACS deposits of \textbf{(b)} 500 \celsius{}, \textbf{(c)} 738 \celsius{}, and \textbf{(d)} 950 \celsius{}.]{Series of electron micrographs near the top, middle, and bottom (but above the deposit-substrate interface of the \textbf{(a)} CS (no laser) deposit, and LACS deposits of \textbf{(b)} 500 \celsius{}, \textbf{(c)} 738 \celsius{}, and \textbf{(d)} 950 \celsius{}. White arrows indicate particle to particle interfaces.}
	\label{fig:43407}
\end{figure}



EBSD\ measurements present a clear change in ferrite/martensite grain size with increasing surface temperature (\ref{fig:43408}). As commonly found in many cold sprayed microstructures, the 4340 steel without laser heating had a small ferrite grain size. In this case, the grain size of the deposit is smaller than the grain sizes found in the feedstock powder particles. It should be noted that standard EBSD pattern indexing routines cannot reliably distinguish between ferrite and martensite, so we will refer to the grain size, or crystallite size, as being from the BCC ferrite phase.  LACS of 4340 steel at 500 \celsius{} was performed with a slightly higher substrate temperature than CS alone, and it produced a slightly larger grain size. The grain shapes for the 500 \celsius{} LACS deposit was also more equiaxed than those in the CS deposit (compare \ref{fig:43408}(c) to \ref{fig:43408}(b)), where the CS deposit had grains that were elongated in shape. The LACS deposition at 738 \celsius{} did not result in a measurable shift in the grain size distribution, \ref{fig:43408}(f). At a surface temperature of 950 \celsius{}, considerable grain growth was observed with the median grain size being four times the size of the CS (no laser) deposit. No retained austenite was observed in any of these deposits. 


\begin{figure}
	\centering
	\includegraphics[width=6.27in]{fig/4340/Picture8.png}
	\caption[Inverse pole figures (y-orientation, or in this case, spray direction) of electron back scattered diffraction maps of different magnifications including: LACS deposits.]{Inverse pole figures (y-orientation, or in this case, spray direction) of electron back scattered diffraction maps of different magnifications including: LACS deposits with surface temperatures of \textbf{(a)} powder (the colored direction legends is applicable to all of the micrographs shown), \textbf{(b) }cold spray without the laser, and LACS with surface temperatures at \textbf{(c)} 500 \celsius{} \textbf{(d)} 738 \celsius{} \textbf{(e)} 950 \celsius{}. White lines indicate low angle grain boundaries with misorientation angles between 2 $ ^{\circ} $  and 10 $ ^{\circ} $. Black lines indicate high angle grain boundaries with misorientation angles greater than 10 $ ^{\circ} $. \textbf{(f)} Their cumulative area fraction as a function of grain diameter.}
	\label{fig:43408}
\end{figure}


The\ appearance of martensite, as shown by BSE micrographs in \ref{fig:43409}, is found at increased surface temperatures. There is very little difference in microstructure from the no-laser cold sprayed condition and LACS with a 500 \celsius{} surface temperature. Below the transformation temperature, there is no phase change.  At 738 \celsius{}, a different polygonal phase is noticeable pointed out by the white arrows in \ref{fig:43409}c. The 950 \celsius{} condition, \ref{fig:43409}d, show large quantities of martensite known through their distinct, lath-shaped structures in the micrograph. As the temperatures cross the initial transformation temperature and the final transformation temperature, different phase structures were formed.


\begin{figure}
	\centering
	\includegraphics[width=6.27in]{fig/4340/Picture9.png}
	\caption{Back-scattered electron micrographs of \textbf{(a)} CS (no laser) and LACS deposits at surface temperatures of \textbf{(b)} 500 \celsius{}, \textbf{(c)} 738 \celsius{}, and \textbf{(d)} 950 \celsius{}.}
	\label{fig:43409}
\end{figure}


\subsection*{Hardness Measurements}

The microhardness of the deposited 4340 steel material evolved in a complex fashion with increasing surface temperature \ref{fig:434010}). The average hardness for the CS deposited sample was 561 HV with the lowest measured hardness (466 HV) for the 500 \celsius{} surface temperature. Above 500 \celsius{}, the hardness increased back to the originally measured hardness of the CS deposited sample. At highest surface temperature of 950 \celsius{}, the microhardness tended towards a more normal hardness distribution. Changing the cold spray gas pressure and temperature also affected the microhardness. At 738 \celsius{}, lowering the gas temperature and pressure had a notable increase in the average microhardness from 532 HV to 592 HV, and also significantly increased the distribution in hardness values. This condition not shown in the figure (738 \celsius{} surface temperature with gas pressure 3.1 MPa and gas temperature 475 \celsius{}) exhibited the largest range in the hardness measurements.


\begin{figure}
	\centering
	\includegraphics[width=6.27in]{fig/4340/Picture10.png}
	\caption{Measured microhardness of CS and LACS deposits at different gas conditions and surface temperatures.}
	\label{fig:434010}
\end{figure}


\section*{Discussion}

\subsection*{Effects of Gas Conditions and Substrate Temperature on Deposition Efficiency}


An increase in particle velocity leads to higher deposition efficiency. If the particle velocity is too low, particles will ricochet off the surface instead of adhering to it \cite{RN1246}. The same principle applies to 4340 steel, and the data in \ref{fig:43404}, provides some initial information about the magnitude of the critical velocity for 4340 steel. \ref{fig:434011}a models gas speed as a function of distance travelled through a de Laval nozzle of the dimensions used in this paper’s cold spray experiment (200 mm length, 2 mm throat diameter, and 6.3 mm nozzle opening diameter). The exit velocity of a 33 $ \mu $m diameter particle is calculated using one-dimensional, isentropic flow of helium gas as developed by Dykhuizen et al. \cite{RN1248}. These calculations show that the 475 \celsius{}, 3.10 MPa gas condition creates a particle speed at the nozzle exit of 1100 m/s. The deposition efficiency of CS at gas temperature and pressure of 550 \celsius{} and 4.14 MPa was 47.9 $\%$ . LACS with a lower gas temperature and pressure 475 \celsius{} and 3.10 MPa\textbf{ }and a laser irradiation causing a surface temperature of 738 \celsius{} has as deposition efficiency of 45.6 $\%$ , only 2.3 $\%$  lower than the recorded CS deposition efficiency. Deposition efficiency of LACS at a gas temperature and pressure of 550 \celsius{} and 4.14 MPa with a surface temperature of 738 \celsius{} measured at 70.7 $\%$ , an increase of 25 $\%$  more than the lower gas temperature and pressure condition at the same surface temperature.


\begin{figure}
	\centering
	\includegraphics[width=6.27in]{fig/4340/Picture11.png}
	\caption{Particle velocity as a function of nozzle length based on different gas temperatures and pressures.}
	\label{fig:434011}
\end{figure}



LACS of 4340 steel exhibits a similar dependence on the surface temperature as LACS experiments reported for other alloy types. In the case of Stellite-6 \cite{RN135}, ODS materials \cite{RN383}, and this 4340 steel study, the deposition efficiencies were measured by differences in mass.  In the case of Ti \cite{RN173} and Cu-based cermets \cite{RN3366}, the change in deposition rate was reported in the literature using deposit. The increase in deposition efficiency for 4340 steel agrees well with the increases reported for the Fe-based ODS alloy and the Cu-based cermet. The data in the literature for LACS of Ti and Stellite-6 report a larger, more temperature sensitive deposition efficiency increase. The study concerning Stellite-6 reported a much smaller but much higher range of homologous temperature compared with other studies, 0.85-0.91 of the melting point of the alloy. There are several possible differences in the response of cold spray deposition to the laser. For example, the titanium and the steels each have important solid state phase transformations which lie within the temperature ranges used in the respective studies. For Stellite and copper, the surface temperatures reported do not cross the temperatures required for any phase transformation. In addition, the velocity distributions of the particles sprayed in each study are very likely different due to differences in nozzle geometries and gas temperatures and pressures. Therefore, there is currently insufficient information in the reported literature to make direct, quantitative comparisons about how particular surface temperatures impact cold spray deposition for different alloys.  In the future, direct comparisons with normalized experimental conditions would provide very useful information about how different crystal structures and alloy types respond to LACS. 


\subsection*{Deposition Mechanisms of LACS}
\label{Deposition Mechanisms of LACS}

The increased surface temperature changes the mechanism by which the particle deforms during deposition. In cold spray, the particle undergoes severe plastic deformation in order to bond with the substrate material. While the very surface of the substrate may experience some grain refinement, it is shown through subsequent characterization that the bottom of the impacting particles flatten and greatly deform with the substrate only partially deforming. For CS, the relative difference in hardness between the particle and the substrate can change the particle-substrate interaction from particle flattening against a hard substrate, to substrate penetration with a softer substrate/harder particle \cite{RN3383}. More pronounced cases of particle-substrate hardness differences are found in LACS, where the substrate deforms while the bottom of the particle is less deformed than in non-laser CS depositions. 


\ref{fig:43406}d shows the clearest example of this phenomenon where the particle impacts the mild steel with the surface temperature at 950 \celsius{}. Here the particles deeply implant themselves into the substrate. Another example may be seen in the 4340 steel powder particles impacting the newly created 4340 steel layer in \ref{fig:43407}. The micrographs of the CS condition show the bottom of the particles to be completely deformed as seen in most CS deposits. With the surface temperature at 500 \celsius{}, the particles still show some deformation. At the 738 \celsius{} and 950 \celsius{} conditions, the interfaces reveal that the round bottoms of the particles are still present, and the tops of the particles are heavily deformed. This deformation at the top of the particles is explained by the thermal softening of the surface being deposited. While the particles briefly pass through the laser, the particles are moving too fast to be meaningfully heated and softened during their transit to the substrate. Thus, the particles (even on the same material type) have a hardness value that is higher than the thermally-softened substrate. 


\subsection*{Oxide development During LACS of Steel}


As additional heat is used in LACS of steel, an increased oxide content is found between the particle and layer interfaces. The optical micrographs (\ref{fig:43405}) at the interface demonstrate dark regions near the particle-particle interface that becomes ever more prevalent at the deposit-substrate interface. BSE micrographs in \ref{fig:43407} show black regions with dark grey regions that surround the black regions present in between the particles. These black regions and dark grey regions possess varying amounts of oxygen, where the increase in oxygen decreases the density and average atomic number of the material, thus reducing the BSE signal strength. This oxygen content was verified through x-ray EDS line scans and map scans (\ref{fig:434012}). 

There is an increasing trend of oxygen content in the deposits with the increase in surface temperature and an increasing trend of oxygen towards the deposit-substrate interface as seen from \ref{fig:43407}. Other than the helium gas flow from the cold spray nozzle, the experiment is kept under normal atmosphere conditions allowing ambient oxygen to reach the deposit. As mentioned in \ref{Sample Creation}, the material was cold spray deposited with six layers. The time it takes to raster a layer is approximately 1 minute and 5 seconds. The total time of deposition and laser irradiation of the metal surface for a six-layer deposit is 6 minutes and 30 seconds. While the temperature of the entire substrate is not as high as the temperature right under the laser spot, the first layer is subjected to some elevated temperatures for approximately 5 minutes and 25 seconds longer than the final layer that is deposited. Although the specific deposit locations cool after the laser irradiation, the deposit is kept at a high enough temperature to increase oxidation.


\begin{figure}
	\centering
	\includegraphics[width=6.27in]{fig/4340/Picture12.png}
	\caption[Electron micrographs of LACS of 4340 steel at a surface temperature of 950 \celsius{}.]{Electron micrographs of LACS of 4340 steel at a surface temperature of 950 \celsius{}. \textbf{(a)} The darker regions in the zoomed out figure are \textbf{(b) }zoomed in. A line scan across the particle interface was performed and \textbf{(c)} the oxygen/metal ratio is plotted. The numbered points in (b) correspond with the numbered peaks in (c). \textbf{(d)} An EDS map scan of a darker grey region in was performed showing \textbf{(e)} Fe and \textbf{(f)} O.}
	\label{fig:434012}
\end{figure}


\subsection*{Microstructural and Hardness Relationships}

The change in deposit hardness with varied surface temperature stems from the evolution of the strengthening mechanisms in the deposited 4340 steel. As shown in \ref{fig:434010}, the hardness of the CS deposition has a median of 568.6 HV. The LACS deposit with a substrate temperature of 500 \celsius{} was lower (476 HV) than the CS deposit and then returned to the original CS hardness as the surface temperature increased. The change in hardness as a function of increasing temperature may be explained through microstructural changes. The EBSD grain sizes, shown in \ref{fig:43408}, clearly tracks how the grains evolved with increasing temperature. The no-laser CS deposit had grains that were smaller than the original starting powder. This reduction in grain size is from the powders experiencing severe deformation with corresponding recrystallization. At 500 $ ^{\circ} $ C surface temperature, there is still some evident recrystallization observed by the grain sizes being smaller than the original starting powder. Nevertheless, the elevated temperature also facilitated some grain growth. According to the Hall-Petch effect, with its connection between strength and hardness, larger grain materials will have a lower hardness. Other microstructure effects, such as phase transformation should not be contributing at this condition. At 500 \celsius{}, the 4340 steel will not experience any ferrite-to-austenite transformation. Additionally, since some martensite was found present in the starting powder material, a heat treatment (by the laser) would temper that remnant martensite present in the material after deposition. Collectively, this leaves the material softer. 


The 738 \celsius{} surface temperature LACS condition is in between the nominal A\textsubscript{c1} and A\textsubscript{c3} temperatures. Here, a phase transformation would be present with some phase fraction of austenite being formed at the 738 \celsius{} which would then transition to martensite upon cooling. \ref{fig:43409}c shows this mix phase content in the microstructure. The white arrows in the figure reveal a polygonal phase mixed into the steel. Despite the further increase in grain growth that occurred with this increase in temperature, the changed microstructure has a higher hardness than the 500 \celsius{} surface condition because of the increased fraction of martensite that would form from cooling from a mixed ferrite and austenite phase field.


When 4340 steel is at 950 \celsius{}, it is expected to be completely austenite. As it cools down, a large portion of the austenite will convert into martensite. This is shown in \ref{fig:43409}d where large amounts of martensitic laths are clearly present. EBSD micrographs reveal that these grain sizes have\ increased by 4X from the non-laser CS deposit. This alone would suggest a significant drop in hardness; however, the hardness is maintained because of the high fraction volume of martensite.  Future efforts will quantify the amount of martensite and the prior austenite grain size for each of these conditions.


\section*{Conclusions}


This study is an initial look into cold spray and laser assisted cold spray of AISI 4340 steel. The following conclusions were drawn:


\begin{enumerate}
	\item Laser assisted cold spray of 4340 steel produces fully consolidated deposits with a clean substrate-deposit interface \textit{and} low porosity. Higher surface temperatures increased deposition efficiency and can make up for sub-optimal gas conditions. However, the combination of good gas conditions and high surface temperatures substantially increases deposition efficiency.
	
	\item The mechanism of deposition changed with increased surface temperature. In cold spray, the particle undergoes severe plastic deformation to achieve deposition. In laser assisted cold spray, the particle deforms less, and the substrate deforms more with increased surface temperatures.
	
	\item High-temperature laser assisted cold spray deposition using the current configuration leads to oxidation between steel particle interfaces, particularly in prolonged sprays. In practical applications, an inert environment may be required if laser assisted cold spray of steel is performed.
	
	\item Microstructure and mechanical properties are tunable with laser assisted cold spray by changing surface temperatures. Hardness of a deposit sprayed at an elevated temperature under the austenite transformation temperature is decreased due to an increase in grain size. Above the austenite transformation temperature, hardness increases despite grain growth because of martensite formation during cooling.
\end{enumerate}


