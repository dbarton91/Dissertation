\chapter{THE INFLUENCE OF ISOCONCENTRATION SURFACE SELECTION IN QUANTITATIVE OUTPUTS FROM PROXIMITY HISTOGRAMS}


Note: This chapter is the contents of the paper by the same name published in Microscopy and Microanalysis \cite{RN1023}.



\section*{Abstract}

	Isoconcentration surfaces are commonly used to delineate phases in atom probe datasets. These surfaces then provide the spatial and compositional reference for proximity histograms, the number density of particles, and the volume fraction of particles within a multiphase system. This paper discusses the influence of the isoconcentration surface selection value on these quantitative outputs using a simple oxide dispersive strengthened alloy, Fe\textsubscript{91}Ni\textsubscript{8}Zr\textsubscript{1}, as the case system. Here, zirconium reacted with intrinsic oxygen impurities in a consolidated ball-milled powder to precipitate nanoscale zirconia particles. The zirconia particles were identified by varying the Zr-isoconcentration values as well as by the maximum separation data mining method. The associated outputs mentioned above are elaborated upon in reference to the variation in this Zr isosurface value. Considering the dataset as a whole, a 10.5 at. $\%$ Zr isosurface provided a compositional inflection point for Zr between the particles and matrix on the proximity histogram; however, this value was unable to delineate all of the secondary oxide particles identified using the maximum separation method. Consequently, variations in the number density and volume fraction was observed as the Zr isovalue was changed to capture these particles resulting in a loss of the compositional accuracy. This highlighted the need for particle-by-particle analysis. 






\section*{Introduction}
	Solute cluster analysis is a fundamental use of data mining in APT aiding in the analysis of features such as nucleation sites, site-specific segregation, dispersants, dislocations, and precipitates. In many of these cases, isoconcentration surfaces have been employed to determine cluster compositions and generate particle size, density, inter-particle spacing, and composition \cite{RN410,RN740,RN2685}. From these isoconcentration surfaces, proximity histograms (or proxigrams) can be created to provide compositional profiles normal to these surfaces \cite{RN1686,RN1009,RN2679}, with a major advantage of a proxigram profile being able to accommodate curved surfaces. When creating this isosurface, by default, proxigrams obtain average compositional information with respect to positions on either side of the isoconcentration surface, with the isosurface spatial location being designated at the ‘zero’ distance position on the proxigram. For example, if an isoconcentration surface delineates a precipitate, the proxigram can yield a profile of the composition inside the precipitate (defined by the surface) and outside the precipitate, which is the matrix. Thus, the selection of the isoconcentration surfaces are commonly used to delineate phases in atom probe datasets. 
	
	APT can provide at or very near atomic spatial resolution data; however, limitations in detection efficiency combined with artifacts from field evaporation show or exaggerate gradients in proxigram as it moves across an interface between two phases. The selection of the isoconcentration surface value can then have a profound impact on the outputted proxigram because the isosurface will be placed along a specific location with respect to this gradient \cite{RN371,RN2603}. Even though such changes in the isoconcentration surface value will change the relative reference point, Yoon et al. have reported that the shape of the proxigram itself will not change \cite{RN683}. Hence, more often than not, most atom probe papers simply then report the isoconcentration value and do not discuss how that value was selected. These values are typically selected by a user-defined visual inspection of how well the surface delineated the feature of interest within the material. Though the proxigram shape may not change, Yoon et al. also commented in the same paper that isosurface shifts will change the reported concentration values at the interface. Thus, user-defined inspections may generate a visually pleasing image, but it may not necessarily provide the most accurate surface needed for the proxigram’s compositional analysis. Furthermore, the volume the isosurface delineates will also impact volume fraction and number density outputs. Since the isosurface selection value plays such a paramount role in the relative compositions and other quantitative outputs that are reported from its surface, understanding how this value is selected for subsequent data analysis is critical. 
	
	For example, in a low-partitioning system, identifying a procedure for isosurface selection is essential for reproducibility. To address this need, Hornbuckle et al. reported a procedure to determine the correct isoconcentration surface value in such a high-solute, low-partitioning alloy \cite{RN214}. In their study, a shape memory alloy was chosen because its functional properties are very sensitive to modest changes in composition and accurate methods for compositional determination were required. As discussed in that paper, the ability to clearly delineate the precipitates was troublesome because the low partitioning of the solute did not readily reveal the precipitates in the atom maps. The high concentration of the solute then inhibited solute clustering methods for precipitate identification. As such, a procedure was outlined in locating the solute inflection point along the compositional profile of the proxigram from which accurate compositions of the precipitates were found and provided consistency in the analysis between different datasets. 
	
	Here, the authors provide a complementary study to the work to Hornbuckle et al. \cite{RN214} by applying the same procedure, but now in a low-solute and high-partitioning alloy where the precipitates are readily identified in an atom map. However, in this new study, we investigate how the selected isoconcentration value in such a visually obvious atom map influences the outputted data from the selected isoconcentration value. Specifically, we report how the particles’ composition, number density, and volume fraction change with the isovalue using a simple ODS alloy created from Fe\textsubscript{91}Ni\textsubscript{8}Zr\textsubscript{1} (at.\%) as our case study \cite{RN267}. The aim of this new investigation is to provide the reader an understanding of how different isoconcentration surface values quantitatively influence the associated statistics on particle information that can be data mined from APT datasets.

\section*{Experimental Procedure}				
	A powder mixture of Fe\textsubscript{91}Ni\textsubscript{8}Zr\textsubscript{1} was ball milled from elemental powders and then consolidated by equal channel angular extrusion (ECAE) at 1000\celsius{} , following the process found in \cite{RN740}. The extruded alloy was deformed at room temperature at a strain rate of 10\textsuperscript{3}s\textsuperscript{-1} using a Kolsky-Hopkinson bar test \cite{RN896}. From this sample, the atom probe specimens were extracted. The requisite needle shape geometry needed for atom probe field evaporation was fabricated by a FIB lift-out procedure similar to \cite{RN347} using a Thermo Fischer Quanta Dual Beam FIB (Hillsboro, OR, USA)  and/or a Lyra Tescan FIB (Warrendale, PA, USA). The tips were sharpened through annular milling at 30 keV with a 5 keV clean-up step to remove surface damage created by the ion beam during the sharpening process. The samples were field evaporated in a Cameca LEAP 5000XS (Madison, WI, USA). Specimens were maintained at 40K with the field evaporation assisted by a 355 nm wavelength ultraviolet laser pulsing at 100 pJ with a dynamic pulse frequency between 300-600 kHz. Detection rates were 1.0 \% events/pulse. The collected data were then reconstructed on the Cameca IVAS later in this paper, simulated atom probe datasets were also generated and done so using a Python 3.6 platform. For both experimental and simulated data, the voxel size was kept at 1 nm x 1 nm x 1.5nm and the delocalization settings were kept at 3.0 nm in the IVAS reconstructions for direct comparison purposes \cite{RN210}. The compositions were reported from the respective proxigrams created by the various Zr isoconcentration surfaces while the number density is simply the number of particles captured within the volume of the tip, and the volume fraction being the collective volume of all the particles normalized to the volume of the tip. 

\section*{Results and Discussion}
	\subsection*{Qualitative Clustering: Visual images for identifying particles}
		\ref{fig:Iso1} shows the atom maps of Zr, ZrN, ZrO, ZrO\textsubscript{2}, and O with the overall composition of the alloy given in \ref{tab:Iso1}. Partitioning behavior in all of these collected species is evident in the reconstruction, with a distribution of particle sizes and qualitatively different compositions in each of these particles evident by the presence of some particles having one but not another element or complex element type in the same spatial location. From these maps, we can conclude that the alloy exhibits a high degree of chemical partitioning in the form of small scale oxide-based dispersions. Using these maps, we will now explore the isoconcentration selection process to delineate these different features from the matrix and their associated quantitative outputs of composition, number density and volume fraction.
	
		\begin{figure}
			\centering
			\includegraphics[width=6.27in]{fig/Isosurfaces/Figure1.png}
			\caption{Selected APT reconstruction atom maps for Fe\textsubscript{91}Ni\textsubscript{8}Zr\textsubscript{1}}
			\label{fig:Iso1}
		\end{figure}
		
		\begin{table}[]
			\caption{Bulk composition from the atom probe data set.}
			\input{tab/Isosurfaces/Isosurfacetab1.tex}
			\label{tab:Iso1}
		\end{table}

	\subsection*{Identifying Clusters with Isoconcentration Surfaces: Using Proximity Histograms}
	\label{Identifying Clusters with Isoconcentration Surfaces: Using Proximity Histograms}
		Recognizing that multiple elements or ion types have partitioned (\ref{fig:Iso1}), we select one of those species to create an isoconcentration value. For this paper, we will use Zr, as it is an intentionally alloyed solute addition and has reacted with the impurities creating both oxide and nitride particles. Using the procedure outlined by \cite{RN214}, an isoconcentration value of 10.5 at. \% Zr was identified for the dataset (as a whole) where the Zr composition had its inflection point at the isosurface value between the particles and the matrix, \ref{fig:Iso2}. At this location, \cite{RN214} reported the highest accuracy for the matrix and precipitate composition. Note that the Zr concentration at this point is 18.3 at. \% Zr. The difference between the isosurface value and the composition noted at the surface is contributed to how each method counts the atoms. In the isovalue method, voxels of the same compositional concentration are linked and averaged creating the surface; whereas, for the composition profile, the proxigram uses the individual binned composition to create the one-dimensional concentration profile. In this study, the differences in these two values, as compared to results as seen in Hornbuckle et al. \cite{RN214}, are contributed to the much more pronounced partitioning behavior of the solute.

		\begin{figure}
			\centering
			\includegraphics[width=6.27in]{fig/Isosurfaces/Figure2.png}
			\caption[Atom probe of Fe\textsubscript{91}Ni\textsubscript{8}Zr\textsubscript{1}.]{Fe\textsubscript{91}Ni\textsubscript{8}Zr\textsubscript{1}. \textbf{(a)} An atom map with two delineated particles (arrows) created by a
				10.5 at. \% Zr isoconcentration surface. \textbf{(b)} The proxigram created from the two particles with the associated composition error having a $\sigma-3${} confidence for the high
				concentration Fe, Zr, and O species. Note that the inflection point of the proximity
				histogram of the Zr solute is at the isosurface interface. \textbf{(c)}: The proxigram created
				from the two particles with the associated composition error having a $\sigma$-3 confidence for the lower concentration Ni, C, N, and Cr species.}
			\label{fig:Iso2}
		\end{figure}

		We note that the oxygen content in these particles are lower than the ideal ZrO\textsubscript{2} stoichiometry. This discrepancy has been seen in \cite{RN368} and explained as a result of the complex and preferential field evaporation events from zirconia \cite{RN807}. Regardless of this specific issue in terms of the exact composition, the inflection point remains a consistent means by which the composition can be determined relative to the isosurface location. Moreover, we can then compare how the relative composition of the particle now changes as a function of the isoconcentration selection.
		
		Using this 10.5 at.\% Zr isosurface value, we have highlighted two particles in the reconstruction by the designated arrows in \ref{fig:Iso2}. These two particles, which are also apparent in the atom maps of \ref{fig:Iso1}, will serve as reference particles for comparisons to all other Zr isosurface selected values discussed in this paper. These two particles were chosen because there would be no controversy as to their existence in the dataset. Furthermore, since both particles are large in size, concerns of trajectory aberrations that have been noted in smaller particles \cite{RN2620} would be mitigated for comparison purposes. 
		
		Upon closer inspection of comparing \ref{fig:Iso2} to \ref{fig:Iso1}, it is obvious that a 10.5 at. \% Zr isosurface has not captured all of the particles present. The inability to delineate all of the particles by this particular isosurface value is linked to compositional variations between the particles. This could be expected as particles nucleate and grow from clusters to precipitates and their compositions can change as well as local magnification effects in the reconstruction of small particles that contribute to compositional inaccuracies \cite{RN806}. To delineate these smaller particles, another Zr isoconcentration value must then be selected. If that value is used from the entire dataset, as done above, it will shift the isosurface interface used for the two identified particles in \ref{fig:Iso2} along its compositional gradient; or in other words, the isosurface is no longer at the Zr compositional inflection point for these particular particles which has previously been identified and discussed to yield the best compositional accuracy. Though one can select these two particles not to be included in any further Zr isosurface value changes within the dataset, one can now appreciate the potentially arduous process for identifying the inflection point for each and every particle. As will be done in the proceeding analysis, the individual selection of a subset of particles will be conducted to study isosurface selection values on a particle-by-particle basis to compare to this single, global value for all the particles. But, at this point of the discussion, we will still allow these two reference particles not to be fixed at this identified 10.5 at. \% Zr isovalue but to vary with all other particles as the isovalue is changed to capture more of the particles seen in \ref{fig:Iso1}. Through these changes, the averaging effect on composition, number density, and volume fraction will then be quantified. 
		
		\ref{fig:Iso3} is a set of images created from 1 to 20 at. \% Zr isoconcentration surfaces at three different $\sigma${} values. Sigma is an inverse square function for the number of atoms sampled to determine the allowable isosurfaces \cite{RN1009}. As both the Zr isoconcentration surface values and confidence $\sigma${} increases, the size of the particles decreases along with the number of isoconcentration surfaces. This is quantitatively plotted in \ref{fig:Iso4}. For the lowest Zr isoconcentration surfaces, the difference between the number density and the sigma values is the largest, which was nearly 2x for the 1 at. \% Zr isosurface sample, \ref{fig:Iso4}(a). However, as the isoconcentration surface value increased, the differences between the sigma values narrow (but never overlap). The corresponding changes in the composition for the particles are shown in \ref{fig:Iso4}(b). Of the elements, Fe, Ni, Zr, and O appeared to have the greatest variation on compositional differences between each other as a function of the selected isosurface and sigma values. The tabulated values from \ref{fig:Iso4}(b) are provided in \ref{tab:Iso2}. The low concentration species C, N, and Cr ($<$ $\sim$1 at.\%) were rather invariant to these changes. At the Zr isoconcentration values greater than 10 at. \%, the differences in the major elements (i.e., those $>$ 1 at. \%) appeared minimal with their mean value differences and associated standard deviations overlapping. One would then assume that an isoconcentration value of $\sim$ 10 at. \% Zr (or greater) would be ideal, but, as already shown in \ref{fig:Iso3}, these values are unable to delineate the majority of the other particles in the dataset. 

		\begin{figure}
			\centering
			\includegraphics[width=5.27in]{fig/Isosurfaces/Figure3.png}
			\caption[Series of APT reconstruction figures concerning at. \% Zr isoconcentration surface value and confidence $\sigma${} values for Fe\textsubscript{91}Ni\textsubscript{8}Zr\textsubscript{1}.]{Series of APT reconstruction figures concerning at. \% Zr isoconcentration surface value and confidence $\sigma${} values for Fe\textsubscript{91}Ni\textsubscript{8}Zr\textsubscript{1}. Data are more sensitive to confidence values at lower isoconcentration surface values than at higher
				isoconcentration surface values.}
			\label{fig:Iso3}
		\end{figure}

		\begin{figure}
			\centering
			\includegraphics[width=6.27in]{fig/Isosurfaces/Figure4.png}
			\caption[Particle number density, concentration, and volume fraction as a function of Zr isoconcentration surface.]{\textbf{(a)} Particle number density as a function of Zr isoconcentration surface. The three different lines are statistical confidences at $\sigma-1${}, $\sigma-2${}, and $\sigma-3${} \textbf{(b)}
				Compositions (at. \%) from inside the isoconcentration surface (particle) as a function of Zr isoconcentration value. \textbf{(c)} Volume fraction as a function of Zr isoconcentration value ($\sigma-3${}).}
			\label{fig:Iso4}
		\end{figure}

		\begin{table}[]
			\caption{Number density of the interface derived from Zr isoconcentration surface at different at. \% and confidence $\sigma${}.}
			\input{tab/Isosurfaces/Isosurfacetab2.tex}
			\label{tab:Iso2}
		\end{table}
		
		Though the composition of the particles identified in \ref{fig:Iso2} and seen in \ref{fig:Iso3} at the highest Zr isosurface values exhibited nearly equivalent particle compositions (\ref{tab:Iso2}) the number density is notably different, with the 10.5 at.\% Zr isosurface being 1.47E22 counts/m\textsuperscript{3} compared to 20 at. \% Zr’s 2.68E21 counts/m\textsuperscript{3}. The volume fraction of the particles also decreased by more than 25\% with each increase in the isosurface Zr percentage values discussed here. In just using these identified particles, where the composition was approximately the same in using isosurfaces at 10.5 at.\% Zr to 20 at.\% Zr, the output number density values can be and are different between each other. When the isoconcentration surface value was reduced from these values, the presence of more particles becomes evident, \ref{fig:Iso3}, but the changes in particle composition (as well as number density and volume fraction) becomes even more varied (\ref{tab:Iso2}).
		 
		Looking at \ref{fig:Iso3}, the 1.0 at. \% Zr isosurface for the lowest $\sigma${} value, is visually inaccurate when compared to the atom maps of \ref{fig:Iso1}. Thus, users would obviously discredit such a value. However, the 5.0 at. \% Zr rendering, at least qualitatively, appears more representative to the partitioning seen in the atom maps in \ref{fig:Iso1}, but its selection reveals notable compositional, number density, and volume fraction differences from the upper isosurface values discussed above in \ref{tab:Iso2}. In \ref{Artificial Particle Identification by Lower Bound Isoconcentration Surface Values: Random Dataset Analysis} we will further discuss the effects of lower bound isosurface values for these quantitative outputs. 
		
		What can be learned from this simple isosurface scanning of the particles over different values is that the quantitative outputs readily change and that the selection of the isosurface value is not necessarily obvious, yet its selection is paramount to the outputs that are being considered. This provides a cautionary example in the isosurface’s use and a need to provide a rationale on how the final isosurface value is selected when discussing outputted values that originate from its selection in APT papers. With the inherent variations as a function of Zr isoconcentration surfaces, we now explore another means in the identification of the isoconcentration surface value that can help in the delineation of the particles.

	\subsection*{Identifying Particles with Isoconcentration Surfaces: Borrowing Help from Maximum Separation Method for Clusters}
	\label{Identifying Particles with Isoconcentration Surfaces: Borrowing Help from Maximum Separation Method for Clusters}

		One means of identifying the smaller particles would be to implement the use of the maximum separation method \cite{RN2642}. The maximum separation method would provide statistical cluster analysis as an independent reference value from which the isosurface value could be adjusted to match the identified particles. This would then provide a guide to the isosurface selection rather than a qualitative matching between images as noted above. From this method for identifying the Zr isosurface, determination on the other data outputs, i.e. composition, could then be made. 
		
		The maximum separation method works by identifying a atom and determines if an equivalent atom is present within a fixed distance, with that distance designated as d\textsubscript{max}. However, equivalent atoms may also be near each other just from statistical randomness. Accordingly, the major challenge for cluster identification is then to identify if such atoms are not random \cite{RN2674,RN2628}. This is achieved by comparing the experimental distribution of the atoms at a designated d\textsubscript{max} for a given minimum set of like atoms that would be present in that distance, denoted N\textsubscript{min}, to an equivalent d\textsubscript{max} for a random distribution. If the two distributions deviate, one can then have confidence that the atoms have clustered. Clusters are then defined primarily by sweeping the d\textsubscript{max} \cite{RN2633,RN1686,RN2620} and N\textsubscript{min} \cite{RN807} parameter space. 
		
		\ref{fig:Iso5} is such a sweep through the dataset as a function of different Zr atom amounts as a function of different d\textsubscript{max} values at an order parameter of one. In general, each curve is similar in shape but different in the number of clusters identified. At the lowest d\textsubscript{max} values, a distinct narrow peak in counts is seen which is followed by a reduction in counts to a minimum value whereupon the curve exhibits a gradual rise and fall in counts as a function of increasing d\textsubscript{max}. The initial peak is associated with clusters whereas the latter (broader) peak is associated with erroneous clusters formed by solute in the matrix because the sampling distance is now so large that it can detect more atoms for a given (larger) distance \cite{RN2674}. Thus, d\textsubscript{max} is a position between these two peaks and its location is typically user defined. Since N\textsubscript{min} represents a cut off in the number of atoms that must be in the cluster, as N\textsubscript{min} increases, a reduction in the number of clusters is observed. We can observe a significant difference in the number of clusters (i.e., counts in \ref{fig:Iso5}) between N\textsubscript{min} = 10 and N\textsubscript{min} = 30, with further increases in N\textsubscript{min} not exhibiting the same extent of differences in the counts via their peak heights. For this alloy, we have concluded that the acceptable d\textsubscript{max} range was between 0.35-0.65 nm with a N\textsubscript{min} between 40 and 50, which would have a difference of about 75 counts. 
		
		\begin{figure}
			\centering
			\includegraphics[width=4.5in]{fig/Isosurfaces/Figure5.png}
			\caption{Cluster analysis using the maximum separation method. A sweep of counts as
				a function of d\textsubscript{max} and N\textsubscript{min}.}
			\label{fig:Iso5}
		\end{figure}
		
		Since the maximum separation method produces a range of user defined values, some ambiguity in the number of counts can exist. Regardless of this particular issue, it does provide guidance to what the relative number of particles should be in the dataset. For example, if we use a d\textsubscript{max} = 0.60 nm and a N\textsubscript{min} = 40, 137 particles or a number density of 1.83E23 counts/m\textsuperscript{3} is generated. For analysis, we applied an envelope and erosion parameter that was the same as the d\textsubscript{max} to improve the accuracy of the data mined clusters. Details on the envelope and erosion parameter can be found in reference \cite{RN807,RN2620}. We now adjust the Zr isoconcentration surface value, as we did in \ref{Identifying Clusters with Isoconcentration Surfaces: Using Proximity Histograms}, to delineate the equivalent number of particles in the dataset that matches this maximum separation method value. This matching of 137 particles was found to occur at a 5.0 at.\% Zr ($\sigma${}isosurface) and is shown in \ref{fig:Iso6}.
		
		\begin{figure}
			\centering
			\includegraphics[width=6.27in]{fig/Isosurfaces/Figure6.png}
			\caption[Maximum separation and isoconcentration methods.]{\textbf{(a)} Maximum separation method identified particles at d\textsubscript{max} = 0.60 nm and N\textsubscript{min} = 40. \textbf{(b)} A Zr 5.0\% ($\sigma-3${}) isoconcentration surface that aligns to the counts
				determined by maximum separation. \textbf{(c)} Proximity histogram of the larger particle at 5.0 at.\% Zr. \textbf{(d)} Proximity histogram of the smaller particle at 5.0 at.\% Zr. \textbf{(e)} Proximity histogram of the smaller particle at 2.0 at.\% Zr.}
			\label{fig:Iso6}
		\end{figure}
		
		If we now decide to use a different d\textsubscript{max} with its associated N\textsubscript{min} value, a different number of particles would be generated; consequently, we would have a different Zr isosurface to match the new number density. For example, if the d\textsubscript{max} is 0.50 nm and N\textsubscript{min} is kept at 40, we mine out 85 particles. More interestingly at d\textsubscript{max} values equal to and less than this 0.5 nm, the largest referenced particle shown in \ref{fig:Iso6}(a) delineated into smaller particles (not shown). Even though this particle is now being separated into smaller volumes, the number of particles at this setting was less because of how the particles are being counted under these d\textsubscript{max} and N\textsubscript{min} values. For the user to generate an equivalent number of 85 particles from the isosurface selected value, a 4.0 at.\% Zr ($\sigma${} isosurface was found for this match. 
		
		Collectively, these results quantitatively demonstrate the care a user must employ in identifying features with these types of analysis methods, whether that be by isosurface selection or cluster data mining. Furthermore, the combined use of these methods can also be used to develop the confidence in the most appropriate values to be selected. In this alloy, we understand that isosurface values less than 10.5 at.\% Zr do not necessarily yield the most accurate (global) composition for all particles because of the aforementioned shift in the isosurface position relative to the Zr compositional inflection point for such particles in the proxigram. To further illustrate this issue, consider the proxigram for the large and small particles identified in \ref{fig:Iso6}(b).
		 
		For the larger particle, the major elements of Ni and Zr (\ref{tab:Iso3}) are not significantly different by the relative shift of the isosurface value relative to the Zr composition’s inflection point, \ref{fig:Iso6}(c). However, this is not the case for the smaller particle where the isosurface is clearly shifted into the particle itself, \ref{fig:Iso6}(d). For the smaller particle, the creation of a 2.0 at.\% Zr isosurface places the isosurface at the inflection point on the proxigram, \ref{fig:Iso6}(e), with the corresponding composition tabulated in \ref{tab:Iso3}. The higher standard error associated with the smaller particle (as compared to the larger particle) is a result of the sampling volume size which is associated with counting statistics that occur with a decrease in particle size. 
		
		\begin{table}[]
			\caption{Individual particle analysis for the two selected particles shown in \ref{fig:Iso6}.}
			\input{tab/Isosurfaces/Isosurfacetab3.tex}
			\label{tab:Iso3}
		\end{table}
		
		What can be concluded from this part of our study is that the 5.0 at.\% Zr isosurface, though matching the number density to the cluster analysis, does not match either the large or small particles’ Zr compositional inflection point. Consequently, this can lead to non-ideal compositional information reporting though we have provided an independent reference means to match the number of particles in our dataset. It is also interesting to note that \ref{tab:Iso3}, which is for a specific particle, can now be compared to the assembled average compositions for all particles at that isosurface value reported in \ref{tab:Iso2}. Significant variation between the individual particle (regardless being either the large or small size) and the assembled average compositions can be seen for all elements. Thus, the selection of the isosurface value, though ideal for number density matching here, has not necessarily translated to sufficient compositional matching for all the particles demonstrating the difficulties of using singular isosurface values for entire dataset descriptions for all types of outputs. 

	\subsection*{Artificial Particle Identification by Lower Bound Isoconcentration Surface Values: Random Dataset Analysis}
	\label{Artificial Particle Identification by Lower Bound Isoconcentration Surface Values: Random Dataset Analysis} 
		Similar to the maximum separation method for helping to identifying parameters for number density reconstructions, random datasets can also be useful in determining a minimum threshold for choosing isoconcentration values. This is particularly important if a data mining method is not used to locate the particles but rather visual inspection via changing the isosurface values.  By using a random dataset, one then knows that no particles should be present in the alloy; therefore, delineation by an isoconcentration surface for a ‘real’ particle should not occur. 
		
		A series of random datasets were created by two methods to identify the lower bound values where artificial delineation becomes present in this alloy. In the first method, all the positions of the atoms stayed the same but the chemical assignments were randomly assigned to those positions. In the second method, the complete spatial randomness (CSR) approach was applied as described by Stephenson et al. \cite{RN1114}. The code used for CSR is found in the supplementary material of this paper. Stephenson et al. commented that even in random experimental data sets, erroneous clustering can still be detected because of local magnification effects created in the reconstruction, but the CSR method’s use of the radial probability distribution function for the placement of the atoms can help mitigate these issues. Nevertheless, the CSR dataset is unable to retain lattice spatial positions, which can exist in an experimental dataset. For those reasons, we applied both randomizing approaches in our analysis.
		 
		\ref{fig:Iso7} is a compilation of isoconcentration surfaces from 0.25 to 1.00 at.\% Zr from the (a) chemically randomized and (b) CSR methods. As the isoconcentration value decreased, the delineation of erroneous features became ever more apparent, even though such small particles are not to be expected. This is an artifact of the isoconcentration value being sufficiently small to link equivalent compositional surfaces together, even in a randomly mixed sample. Note that in some isosurface selections, the $\sigma-2${} and $\sigma-3${} confidence reconstructions were not present because the counting statistics to create such a surface were not met. Based on the analysis of the random data sets, the values before these artificial delineations became apparent would represent the lower bound in the isosurface values that one could possibly consider as reliable. \ref{tab:Iso4} is a tabulation for these lower bound values for all $\sigma${}’s and isoconcentration combinations.
			
		\begin{figure}
			\centering
			\includegraphics[width=6.27in]{fig/Isosurfaces/Figure7.png}
			\caption[A series of figures showing the evolution of isoconcentration surfaces as functions of Zr isoconcentration surfaces and $\sigma${} confidences.]{A series of figures showing the evolution of isoconcentration surfaces as functions of Zr isoconcentration surfaces and $\sigma${} confidences by \textbf{(a)} chemically randomizing the data while keeping the same atom positions and
				\textbf{(b)} CSR. The
				mass spectrum is the same as the original experimental atom probe data and the
				mass-to-charge ratio assignments to positions were randomized.}
			\label{fig:Iso7}
		\end{figure}	
			
		\begin{table}[]
			\caption{Maximum Zr isoconcentration values (with different $\sigma${} confidences that can create interfaces from randomized datasets).}
			\input{tab/Isosurfaces/Isosurfacetab4.tex}
			\label{tab:Iso4}
		\end{table}
		
		Using the minimum values set for $\sigma-1${} and $\sigma-3${} at their assigned lower isoconcentration values of 1.46 at. \% Zr and 0.39 at. \% Zr respectively, the experimental data sets were then reconstructed under these conditions and are shown in \ref{fig:Iso8}(a)-(b). A qualitatively large number of particles are now created, which would be expected from the discussion in \ref{Identifying Particles with Isoconcentration Surfaces: Borrowing Help from Maximum Separation Method for Clusters}. Under these conditions, the number density was found to be 3.75E23 counts/m\textsuperscript{3} for the 1.46 at. \% Zr/$\sigma-1${} and 3.41E23 counts/m\textsuperscript{3} for 0.39 at. \% Zr/$\sigma-3${}. At these lower isoconcentration values, the compositional variation in the particle as compared to the prior (higher) isosurface values is, as to be expected, significantly different as is the number density and volume fraction, \ref{tab:Iso2}. Even though we now have identified a lower bound in the value that could be used, these values are yielding much less reliable results. For example, consider the compositional information. \ref{fig:Iso8}(c) is the location of the 1.46 and 0.39 at. \% Zr isosurfaces relative to the proxigram. Each of these surfaces delineate features above the random threshold criteria above but are clearly not near the inflection point for the global average of all the particles. While the lower bound values can set a confidence value for what could be used to delineate features from a random data set, the compositional accuracy for isovalues that delineate the smallest features are a major concern and such isovalues should be used with caution. 
		
		\begin{figure}
			\centering
			\includegraphics[width=5in]{fig/Isosurfaces/Figure8.png}
			\caption[Isoconcentration surfaces of $\sigma${} and Zr isoconcentration surfaces for the experimental dataset determined by the lower bound limits from the CSR analysis.]{Isoconcentration surfaces of $\sigma${} and Zr isoconcentration surfaces for the experimental dataset determined by the lower bound limits from the CSR analysis.
				Isosurface maps \textbf{(a)} for 1.46 at. \% Zr at $\sigma-1${} and \textbf{(b)} for 0.39 at. \% Zr at $\sigma-3${} \textbf{(c)}
				Proximity histogram from the surfaces shown in a: and b: with the Zr composition plotted.}
			\label{fig:Iso8}
		\end{figure}

	\subsection*{Particle-by-particle quantification} 

		From the collective data analysis, one can conclude that a singular, isoconcentration value is insufficient in generating both an accurate compositional profile for all particles and the correct number density/volume fraction of particles in a singular dataset. One possible method to remedy this discrepancy is to identify the number of particles, for example by a cluster analysis, and then use the isosurface selection to correctly capture that number of particles. In this way, the selected isosurface has properly accounted for the correct number of particles in your system, which can be ambiguous if one simply adjusts the isosurface value randomly to delineate the particles from the matrix. Using the information from \ref{Artificial Particle Identification by Lower Bound Isoconcentration Surface Values: Random Dataset Analysis}, one can even have confidence that the selected isosurface value is not beyond the lower bound for this delineation of these features. With the particles now identified, the user will then systematically adjust each individual isosurface value such that the proper spatial position of the isosurface is at the inflection point on the proxigram for accurate compositional quantification. 
		
		Depending on the number of particles, this process could be quite laborious but would be accurate in both the number of particles and the composition of each particle. To determine the variability of these changes (and even possible trends) to our dataset, we have randomly selected 16 particles shown in \ref{fig:Iso9}a with corresponding data plotted in \ref{fig:Iso9}b with such data also tabulated in the appendix (\ref{tab:Isoapp1}). As can be seen in \ref{fig:Iso9}b, there is no apparent trend in the Zr concentration in the particle and the volume of the particle when each particle’s isosurface is adjusted to have the Zr inflection point at that isosurface value. This further confirms that one would need to individually assess each particle and cannot necessarily make predictions for changes simply based on size, at least for this alloy example. Algorithms which can automatically provide this level of individual assessment would be beneficial and an area for future work. 
		
		\begin{figure}
			\centering
			\includegraphics[width=6.27in]{fig/Isosurfaces/Figure9.png}
			\caption[Isoconcentration surfaces created by looking at individual particles and then adjusting the isoconcentration value until the interface is aligned with the
			inflection point on the particle.]{Isoconcentration surfaces created by looking at individual particles and then adjusting the isoconcentration value until the interface is aligned with the
				inflection point on the particle. The chosen Zr isoconcentration value (at. \%) is shown by colors with a corresponding temperature map on the left. b: Volume and
				composition as a function of isosurface value measured from the individual particles in the figure.}
			\label{fig:Iso9}
		\end{figure}
		
\section*{Conclusions}
	A Fe\textsubscript{91}Ni\textsubscript{8}Zr\textsubscript{1} ODS alloy was used as a case study to show how variations in isoconcentration values delineate particles which can lead to a range of compositions of those particles and varied number densities. Though our case study is specific to this alloy, it does provide relative concepts of how isosurface values in high partitioning, low solute alloys influence quantitative outputs of particle composition, number density, and volume fraction. Specific findings are summarized:
	Ideally, the selection of the isoconcentration surface value for compositional analysis should be at the inflection point of the chosen species that delineates the particle from the matrix. However, in a multi-particle dataset, compositional variations can exist between all particles and the selection of a singular value that resides at the compositional inflection point for all particles may not be possible. Individual particle-by-particle analysis may then be required. Thus, if a singular value is used, as one adjusts the isoconcentration value for the multiple particles, one should exhibit caution (and track) how the compositional variation changes with the isosurface selected values. 
	
	As the isoconcentration value decreased, the number density increased. One means of identifying an isoconcentration surface value for the number density is to compare values that create equivalent surfaces and number density values from cluster analysis methods. Here we used the maximum separation method. However, the isoconcentration value identified for the number density match to the cluster analysis was found not to lie at the inflection point for accurate compositional analysis for either a large or small particle via their proxigrams. Thus, caution in how a single isosurface value is used for each output (number density and particle composition) should be considered.
	
	As the isoconcentration value decreased, erroneous delineations of particles can occur in a sample. Using randomized datasets, lower bound values for these isosurface values that could be used to delineate particles in an experimental dataset were investigated. These lower values, though giving confidence in identifying real particles, are not necessarily reliable in creating surfaces from which accurate compositional information or number density/volume fraction information can be obtained. 

	The collective findings of this paper has shown how different isosurface values are used and create different quantitative outputs. The results provide the reader an understanding of how these values relatively change in terms of upper and lower bound values with respect to different quantitative outputs and the care one must exhibit in the selection of the isosurface for a given output. In this work, it was found that one encompassing isosurface value was inappropriate for all desired outputs. Depending on the desired information, particle-by-particle analysis would be required to ensure the highest accuracy in the reported data. However, for this alloy, the variation in the Zr concentration was shown not to have any obvious correlation with particle volume size which, unfortunately, would not enable a means for predicting possible inaccuracies. If such particle-by-particle analysis is not possible in large particle number datasets, one should then exhibit caution in applying and reporting a common singular value for all reported quantitative outputs from a single dataset. Finally, since the isosurface value itself contributes significantly to the various quantitative outputs of composition, number density and volume fraction, it is encouraged that the rationale on how that value was determined should be provided in APT based papers. 
